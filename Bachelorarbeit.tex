\documentclass[12pt, titlepage]{article}
\usepackage[ngerman]{babel}
\usepackage[utf8]{inputenc}
\usepackage{color}
\usepackage[a4paper, lmargin={3cm}, rmargin={2.5cm}, tmargin={3cm}, bmargin={2cm}]{geometry}
\usepackage{amssymb}
\usepackage{amsthm}
\usepackage{graphicx}
\usepackage{helvet}
\usepackage{microtype}
\usepackage{setspace}
\usepackage{csquotes}
\usepackage{xpatch}
\usepackage{comment}
\usepackage{multirow}
\usepackage{tabularx}
\usepackage{capt-of}
\usepackage[backend=biber, %% Hilfsprogramm "biber" (statt "biblatex" oder "bibtex")
style=authoryear-ibid, %% Zitierstil (siehe Dokumentation)
natbib=true, %% Bereitstellen von natbib-kompatiblen Zitierkommandos
hyperref=false, %% hyperref-Paket verwenden, um Links zu erstellen
ibidpage=true
]{biblatex}
\addbibresource{literatur.bib}

\renewcommand{\familydefault}{\sfdefault}
\setlength\parindent{0pt}


\begin{document}
% Leere Titelseite
\pagenumbering{gobble}
\newpage\null\thispagestyle{empty}\newpage

\begin{titlepage}
    \normalsize{Hochschule Hannover, Fakultät IV: Wirtschaft und Informatik \\
    Bachelorarbeit im Studiengang Wirtschaftsinformatik, Wintersemester 2021/2022} 
    \vspace{1.5cm}
    \sloppy 
    \textbf{\Large{\\Konzeption, Datenmodellierung und prototypischer Aufbau eines Prozess-Tracking-Tools zur Steuerung und Umsetzungsverfolgung einer S/4HANA Transformation im Vorgehensmodell eines IT-Beratungsunternehmens}}
    \vspace{10cm}
    \normalsize{\\Abgabedatum: 08. Februar 2022 \vspace{1cm}\\Lukas Hampel\\Matrikelnummer: 1481025\\Scharnhorststr. 8\\31785 Hameln\vspace{1cm}\\Erstprüfer: Herr Prof. Dr. Raymond Fleck\\Zweitprüfer: Herr Michael Bloß, adesso orange AG}
\end{titlepage}


\section*{Sperrvermerk}
Lorem
\newpage


\pagenumbering{Roman}
\setcounter{page}{3}
\section*{Vorbemerkung}

\newpage

\tableofcontents

\newpage

\section*{Abkürzungsverzeichnis}

\newpage

\section*{Abbildungs-/Tabellenverzeichnis}

\newpage

\section*{Kurzfassung}

\newpage

\pagenumbering{arabic}
\setcounter{page}{1}
\begin{normalsize}
\linespread{1.5}

%Einleitung
\doublespacing
\section{Einleitung}
\subsection{Motivation}
Die SAP SE (fortan, in Abgrenzung zum Produkt, als \glqq{}die\grqq{} SAP bezeichnet) ist der größte Anbieter für Unternehmenssoftware in Europa \footcite[Vgl.][]{sap-about} und hat mit dem Produkt SAP-ERP eine der am weitesten verbreiteten Enterprise-Ressource-Planning (ERP)-Software geschaffen \footcite[Vgl.][]{sap-about}. \\Mit der neusten Generation SAP S/4HANA sollen in den nächsten Jahren die bereits etablierten Versionen SAP R/2 und SAP R/3 sukzessive abgelöst werden, bevor die Unterstützung, in Form von Weiterentwicklungen und Aktualisierungen, durch die SAP bis zum Jahr 2030 vollständig eingestellt wird\footcite[Vgl.][]{sap-support}. Die neuste Generation, SAP S/4HANA, bringt viele neue Funktionen mit sich, unter anderem eine neue Datenbanktechnologie und Cloud-Lösungen, weshalb die Umstellung für die meisten Unternehmen eine große Hürde darstellt, die in der Regel nicht mit den intern vorhandenen Ressourcen bewältigt werden kann. Allerdings bringt die Aktualisierung auf die neuste Generation auch viele Chancen mit sich, um den Aufbau der Systeme und den darin abgebildeten Geschäftsprozesse komplett neu zu denken, da durch die Änderungen und neuen Funktionen üblicherweise viele Prozesse bei der Umstellung überarbeitet werden müssen. Das erleichtert beispielsweise die Abtrennung von historisch gewachsenen Strukturen und die Annäherung bzw. die Etablierung des Industriestandards und dessen Best Practices. Dadurch erreicht man im Anschluss eine Verringerung der Wartungskosten für die Systeme und eine Optimierung und Effizienzsteigerung der Geschäftsprozesse.\\ 
Die SAP setzt in den Bereichen Vertrieb, Service, Betrieb und  Entwicklung ihrer Produkte auf ein breit aufgestelltes Partnerprogramm, in dem Drittunternehmen aufgenommen werden können, um sich für eine Kooperation zu qualifizieren \footcite[Vgl.][]{sap-partner}. Dadurch haben sich viele IT-Beratungsunternehmen auf das Themengebiet SAP spezialisiert und bieten nun auch eine SAP S/4HANA-Transformation für ihre Kunden an. Um eine S/4HANA-Transformation durchzuführen, ist viel Wissen und Erfahrung im Projektmanagement und der Projektorganisation notwendig, vor allem aber auch viel Expertise in den Disziplinen der einzelnen Fachbereiche. Dabei kommen viele unterschiedliche Methodiken und Tools zum Einsatz, die den Projektmitarbeitern die Arbeit erleichtern und ihnen stets eine Übersicht über die bereits geleistete Arbeit und den noch zu erledigenden Aufgaben geben sollen. 

\subsection{Zielsetzung und Vorgehen}
In der hier vorliegenden Bachelorarbeit aus dem Studiengang der Wirtschaftsinformatik soll es um die Konzeption, Datenmodellierung und den prototypischen Aufbau eines Prozess-Tracking-Tools gehen, das im Vorgehensmodell eines IT-Beratungsunternehmens zur SAP S/4HANA-Transformation zum Einsatz kommen soll.\\ Das Tool soll auf dem gesamten Transformationspfad eines S/4HANA-Projekts produktiv zum Einsatz kommen und frühzeitig einen Überblick über alle betroffenen Geschäftsprozesse geben und den aktuellen Transformationsfortschritt dieser, bis zur Fertigstellung des Projekts, erfassen und wiedergeben können. Dadurch soll erreicht werden, dass zu jedem Zeitpunkt in einem Projekt, der aktuelle Transformationsfortschritt eines Geschäftsprozesses begutachtet werden kann. Durch die ganzheitliche Erfassung Prozessen soll erreicht werden, dass keine Details oder Bestandteile außer Acht gelassen werden, wodurch Probleme im Projektverlauf vermieden und die später notwendige Fehlerbehebung auf ein Minimum reduziert werden soll. Das erklärte Ziel ist dabei die Qualitätsverbesserung und die Effizienzsteigerung der S/4HANA-Transformation.\\
In dieser Arbeit soll es in erster Linie um die Konzeptionierung dieses Tools gehen. Dazu wird der Requirements Engineering-Prozess bis zu der Erreichung eines GUI-Prototyps umgesetzt und eine Datenmodellierung in Form von UML-Diagrammen angefertigt. Bis mit der Konzeptionierung jedoch begonnen wird, werden zuerst die theoretischen Grundlagen zu SAP und zur S/4HANA-Transformation beschrieben und im Anschluss der unternehmerische Kontext beschrieben, in dessen die Systemkonzeption stattfindet. Dazu wird das auftraggebende Unternehmen vorgestellt, sein Geschäftsmodell beschrieben und das selbst entwickelte Vorgehensmodell zur S/4HANA-Transformation erklärt. Im Anschluss beginnt der Konzeptionierungsprozess mit der Erhebung des Ist-Zustandes, in der die Problemstellung dargestellt und die aktuell im Einsatz befindliche Lösung beschrieben wird. Danach erfolgt, nach Analyse der verschiedenen Stakeholder, die Spezifikation der Anforderungen, die im Anschluss mithilfe einer objektorientierten Analyse und eines Datenmodells analysiert werden. Ergebnis dieses Requirements Engineering-Prozesses ist eine fachliche Lösung mit einem OOA-Modell (Objektorientiertes-Analyse-Modell) und einem Oberflächen-Prototyp, der die Datenelemente des OOA-Modells auf der Benutzeroberfläche abbildet.\\Nach Beendigung des Analyseprozess folgt zum Schluss das abschließende Fazit mit einem Ausblick. 



%Grundlagen
\section{Grundlagen}
Im folgenden Kapitel werden die theoretischen Grundlagen behandelt, die für das Verständnis dieser Arbeit notwendig sind. Dabei geht es in erster Linie um allgemeine Begrifflichkeiten aus dem wirtschaftlichen Kontext des zu entwickelnden Programms.

\subsection{ERP-Systeme}
ERP ist ein Akronym für den englischen Begriff \glqq{}Enterprise Ressource Planning\grqq{}, also das Planen von Unternehmensressourcen, u.a. in den Bereichen Beschaffung, Produktion, Vertrieb, Personalwirtschaft und Finanzwesen. \footcite[Vgl.][523]{wibuch} Ein ERP-System beschreibt somit eine Software, die Prozesse aus diesen Bereichen in einem Anwendungspaket integriert und die dabei anfallenden Daten in einer zentralen Datenbank abspeichert. Dadurch werden Redundanzen in der Datenhaltung vermieden und die Umsetzung von bereichsübergreifenden Unternehmensprozessen ermöglicht \footcite[Vgl.][523]{wibuch}. ERP-Systeme nutzen in der Regel eine Client-Server-Architektur und sind komponentenorientiert, das heißt, Unternehmen können, je nach Anforderungen ihrer Wertschöpfungsprozesse, die benötigten Komponenten frei wählen. Dadurch ist eine schrittweise Einführung der ERP-Software, über einen längeren Zeitraum, möglich. \footcite[Vgl.][524 f.]{wibuch}

\subsection{SAP}
\subsubsection{Die SAP SE}
Die SAP SE wurde im Jahr 1972 von fünf ehemaligen IBM-Mitarbeitern unter dem Namen \glqq{}\underline{S}ystem\underline{a}nalyse und \underline{P}rogrammentwicklung GbR\grqq{}\footcite[Vgl.][]{think-ing}  mit dem Ziel gegründet, eine Standardanwendungssoftware für die Echtzeitverarbeitung zu entwickeln.  Im Jahr 1973 wurde durch die SAP mit dem \glqq{}System RF\grqq{} das erste Produkt für die Finanzbuchhaltung vorgestellt, was den Grundstein für die erste SAP-Generation \glqq{}SAP R/1\grqq{} legen sollte. Durch die ständigen Weiterentwicklungen wurde das System stets erweitert und fand bei immer mehr Kunden Anklang. 1976 wurde die Gesellschaft bürgerlichen Rechts aufgelöst und in eine GmbH überführt. Im selben Jahr wurde bereits mit nur 25 Mitarbeitern ein Umsatz von 3,81 Mio. DM erzielt. \footcite[Vgl.][]{sap-fruehejahre}\\Im Jahr 1979 folgt schließlich die zweite Produktgeneration \glqq{}SAP R/2\grqq{}, die eine höhere Stabilität mit sich brachte und in weitere Geschäftsbereiche vordrang. In der Generation R/2 waren bereits die Module RF für Finanzbuchhaltung, RK für die Kostenrechnung, RM für Materialwirtschaft, Produktionsplanung und Instandhaltung, RP für die Personalwirtschaft und RV für den Vertrieb verfügbar.\footcite[Vgl.][]{bewerbungsratgeber}\\Im Jahr 1988 wurde die SAP GmbH schließlich in eine Aktiengesellschaft überführt und startete an der Börse Frankfurt sowie in Stuttgart. Im selben Jahr erwirtschaftetet SAP bereits einen Umsatz von 245 Mio. DM und hatte bereits 940 Mitarbeiter. Bereits zu diesem Zeitpunkt war die dritte Generation \glqq{}SAP R/3\grqq{} in Entwicklung, die schließlich im Jahr 1992 erschien und, in Gegensatz zu ihren Vorgängern, die als Mainframe-Anwendungen liefen, auf einer Client-Server-Architektur aufgebaut war. Das führte dazu, dass SAP immer erfolgreicher wurde und auch international immer weiter expandierte, sodass im Jahr 1997 schließlich 1,6 Mrd. DM  Umsatz erwirtschaftet wurden. 1995 begann SAP damit, seine Vertriebsaktivitäten im deutschen Mittelstand auszubauen, da zuvor die Hauptkundenzielgruppe nur größere Unternehmen waren. In den darauffolgenden Jahren startete die SAP zusammen mit Microsoft seine Internetstrategie und setzte mit \glqq{}mySAP.com\grqq{} vermehrt den Fokus auf E-Commerce und E-Business-Lösungen und seit dem Jahr 2007 auch auf Business Intelligence.\footcite[Vgl.][]{sap-fruehejahre} Ab dem Jahr 2009 richtete sich die SAP verstärkt auf die Bereiche der Datenbanktechnologie und Cloud Computing aus, woraus schließlich im Jahr 2011 die Datenbanktechnologie \glqq{}SAP HANA\grqq{} entstand, die vor allem Geschwindigkeitsoptimierungen in der Datenverarbeitung mit sich brachte. 2015 wurde schließlich die vierte, heute noch aktuelle, SAP-Generation \glqq{}SAP S/4HANA\grqq{} vorgestellt, die vollständig auf SAP S/4HANA basiert und eine moderne Benutzeroberfläche mit sich bringt, mit der Anwendungen auch auf mobilen Endgeräten dargestellt werden können. Auch bietet die SAP mit S/4HANA erstmalig Cloud-Lösungen für ihre Kunden, was besonders auf kleine und mittelständische Unternehmen abzielt.\footcite[Vgl.][]{sap-historie}\\ Im Jahr 2020 belief sich der Gesamtumsatz der SAP SE auf 27,338 Mrd. EUR (IFRS), worauf alleine ca. 15 Mrd. EUR auf den Vertrieb von \glqq{}On-premise\grqq{} Softwarelizenzen und -Support zurückzuführen sind und ca. weitere 8 Mrd. EUR auf die Umsätze mit Softwarelizenzen und -Support aus den Cloud-Plattformen zurückgehen. Nach Abzug der operativen Aufwendungen und der Steuern blieben davon 5,238 Mrd. EUR Gewinn (IFRS).\footcite[Vgl.][S. 142]{sap2020-report} 

\begin{figure}[h!]
    \centering
    \includegraphics[scale=1]{Bilder/SAPIntelligentesUnternehmen.png}
    \caption[Das Intelligente Unternehmen]{Das Intelligente Unternehmen (Quelle: \cite[][S. 53]{sap2020-report})}
\end{figure}

Die SAP verfolgt derzeit die Vision ihre Kunden zu einem intelligenten Unternehmen zu entwickeln, in denen die Prinzipien der Innovation, Integration, Agilität und Geschwindigkeit an vorderster Stelle stehen. Außerdem alle Elemente eines Unternehmens verbunden werden und ineinandergreifen. Die Komponenten eines solchen intelligenten Unternehmens sind nach Vorstellungen der SAP ein Geschäftsnetzwerk, das die unternehmensübergreifenden Prozesse miteinander verknüpft, eine Business Process Intelligence, die die Geschäftsprozesse analysiert und optimiert, das Experience Management, das die Daten der Anwender, Kunden und Mitarbeiter analysiert, eine Business Technology Platform, die das Fundament für die Integration und Erweiterung von Anwendungen liefert und dem Kunden Möglichkeiten für künstliche Intelligenz, maschinelles Lernen und Prozessautomatisierung bietet, und einem SAP-Rechenzentrum oder einem Hyperscaler, also einem Infrastructure-as-a-Service Anbieter wie Amazons AWS oder Microsoft Azure.\footcite[Vgl.][S. 53 f.]{sap2020-report}. Dadurch soll das Ziel erreicht werden, langfristig die Abläufe der weltweiten Wirtschaft zu verbessern.


\subsubsection{SAP-ERP}
\label{kap:R3}
SAP ERP, oder auch SAP ECC (SAP ERP Central Component), ist die Weiterentwicklung der dritten Generation des SAP ERP-Systems, \glqq{}SAP R/3\grqq{}, das im Jahr 1992 die zweite Produktgeneration \glqq{}SAP R/2\grqq{} ablöste und im Jahr 2003 in \glqq{}SAP ERP\grqq{} umbenannt wurde.\footcite[Vgl.][]{sap-unterschiede} Der Name setzt sich dabei aus dem \glqq{}R\grqq{} für den Begriff \glqq{}Realtime\grqq{}, also Echtzeit, für die Echtzeitdatenverarbeitung und der \glqq{}3\grqq{} zum einen für die dritte Generation, aber auch für die dreischichtige Architektur, die dem System zugrunde liegt, bestehend aus Datenbank, Anwendungsserver und Client. Die dritte SAP-Generation verfügt dabei über eine zentrale Datenbank, in der alle Daten aus den einzelnen Modulen und den verteilten Anwendungen gesichert werden. SAP ERP bzw. SAP ECC stellt die zentrale Komponente der \glqq{}SAP Business Suite\grqq{} dar, in der noch andere Produkte von SAP erhältlich sind, die auf andere Anwendungsbereiche als ERP abzielen, aber mit denselben Daten arbeiten, zum Beispiel dem CRM (Customer Relationship Management) oder dem SCM (Supply Chain Management).\footcite[Vgl.][]{mindsquare-sap} Durch die unterschiedlich ausgerichteten Systeme können sich die Kunden ihre Systemlandschaft frei zusammenstellen und diese spezifisch an ihr Geschäftsmodell anpassen. Dadurch wird eine noch tiefer gehende Integration von Geschäftsprozessen ermöglicht, da all diese Systeme mit derselben, zentrale Datenbanken arbeiten. Die aktuellste SAP-ERP Version ist das Enhancementpackage 8 für SAP ERP 6.0 und ist im Jahr 2016 erschienen, da \glqq{}SAP R/3\grqq{} seit 2015 durch die neuste Generation \glqq{}SAP S/4HANA\grqq{} abgelöst wurde.\footcite[Vgl.][]{sap-version}\\ SAP bietet für das Grundsystem unterschiedliche Modulen an, die das System erweitern und ebenfalls durch die Kunden frei, nach ihren jeweiligen Anforderungen, zusammengestellt werden können.

\begin{figure}[h!]
    \centering
    \includegraphics[scale=1]{Bilder/sap-module.jpg}
    \caption[Die Module von SAP-ERP]{Die für SAP-ERP erhätlichen Module (Quelle: \cite[][]{sap-module})}
    \label{fig:sapmodule}
\end{figure}

In der Abbildung \ref{fig:sapmodule} sind die für SAP-ERP erhältlichen Module und die dazugehörigen Anwendungen bzw. Systeme abgebildet. Die wichtigsten Module und gleichzeitig den Kern des Systems stellen dabei die Module FI, CO, MM, SD, PP und HCM dar, die zum Teil auch standardmäßig in jeder SAP-Installation vorinstalliert sind.\footcite[Vgl.][S. 8]{sap-für-wp} Auf der rechten Seite der Abbildung \ref{fig:sapmodule} sind im inneren Kreis der äußeren Umrandung der Raute, in Rot, die Module des Rechnungswesens dargestellt, FI für das externe Rechnungswesen und CO für das Controlling, bzw. das interne Rechnungswesen sind dabei am weitesten verbreitet. Auf der linken Seite sind in Grün die Logistikmodule zu sehen, bei denen PP für die Produktion, MM für die Materialwirtschaft, SD für den Vertrieb und PM für die Instandhaltung im Vordergrund stehen. Als dritte Kategorie kommt in den aktuellen Versionen von SAP ERP noch das Modul Human Capital Management (HCM) für die Personalwirtschaft dazu, das das HR-Modul abgelöst hat.\footcite[Vgl.][]{sap-module2} Im äußeren Kreis der äußeren Umrandung der Raute sind in Blau die Module der SAP Business Suite dargestellt und in Orange die zusätzlich erhältlichen SAP-Produkte, für CRM, SCM, etc.\footcite[Vgl.][]{sap-module}
\\Mit der Vorstellung der neusten SAP Generation \glqq{}SAP S/4HANA\grqq{} hat SAP angekündigt, die Unterstützung, in Form von Updates und Fehlerbehebungen, von SAP-ERP nach 10 Jahren, also im Jahr 2025, einzustellen und nur noch die neuste Generation, S/4HANA, zu unterstützen. Aufgrund der weiten Verbreitung von SAP-ERP und einer durch die Abkündigung entfachter Debatte hat die SAP angekündigt, die Kernanwendungen von SAP-ERP noch bis 2027 zu unterstützen und bis 2030, gegen Aufpreis, eine erweiterte Unterstützung anzubieten. Mit dieser Maßnahme möchte die SAP erreichen, dass alle Unternehmen in den nächsten Jahren auf die neuste Generation umsteigen.\footcite[Vgl.][]{sap-support}

\subsubsection{SAP HANA}
\label{kap:HANA}
SAP HANA ist eine sogenannte \glqq{}In-Memory\grqq{}-Datenbanktechnologie, die eigens durch die SAP für ihre Produkte entwickelt wurde, mit der große Datenmengen schnell ausgewertet werden können. Die Abkürzung HANA steht dabei für \glqq{}Hyper Perfomance Analytic Appliance\grqq{}, zu Deutsch etwa \glqq{}Höchstleistungsauswertungsinstrument\grqq{}. Die Technologie wurde bereits im Jahr 2008 von SAP in Zusammenarbeit mit der Universität Stanford und dem Hasso-Plattner-Institut\footnote{Hasso Plattner ist einer der Mitbegründer der SAP} entwickelt. Die Besonderheit von HANA ist, dass es sich dabei um sogenannte \glqq{}In-Memory-Datenbanken\grqq{} und die Inhalte der Datenbank durchgehend im Hauptspeicher (RAM) geladen sind und nicht, wie bei herkömmlichen relationalen Datenbanken, nur der aktuell für die Verarbeitung benötigte Teil vom Dauerspeicher in den Hauptspeicher geladen wird.\footcite[Vgl.][]{was-hana} Dadurch sollen die Zugriffsgeschwindigkeiten bis \glqq{}[...] zu 100.000-mal schneller als bei einer Festplatte[..]\grqq{}\footcite[Vgl.][]{rz10-hana} sein. Das vollständige Laden der Datenbank bewirkt zwar, dass dadurch der Hauptspeicher stark belastet wird und durch die Datenmengen entsprechend viel Kapazität benötigt, jedoch bewirkt das Moore'sche Gesetz, dass durch das Voranschreiten der Technologien alle 18-24 Monate die mögliche Computerleistung verdoppelt wird und gleichzeitig die Preise je Speichereinheit sinken.\footcite[Vgl.][]{mooresches}\\ Da, im Gegensatz zu dem Stückweisen Laden aus dem Datenspeicher, der Zugriff aus dem Hauptspeicher deutlich schneller vonstattengeht, ermöglicht HANA somit eine verbesserte Verarbeitung von großen Datenmengen mit hoher Geschwindigkeit. So ist es Unternehmen möglich, die gesamte Datenbank in Echtzeit zu analysieren und darauf basierende Entscheidungen zu treffen, wodurch Geschäftsprozess beschleunigt und effizienter gemacht werden können. Eine weitere Besonderheit von HANA ist die Spaltenorientierung, da traditionelle, relationale, Datenbanken in der Regel zeilenorientiert arbeiten und die einzelnen Datensätze je Zeile gespeichert werden. Dadurch ist es möglich, zum Beispiel bei Auswertungen, schnell auf alle Dateneinträge eines Datenbankattributs zuzugreifen, da diese zusammen in einer Zeile gespeichert werden. Auch sind in HANA analytische und transaktionale Daten gemeinsam verfügbar und die analytischen Daten werden nicht, wie bei herkömmlichen Datenbanken, vorher repliziert. Dadurch arbeiten Analysen in HANA stets mit den aktuellen, transaktionalen, Datensätzen. Um die Schreibvorgänge zu beschleunigen, gibt es in HANA einen Buffer, der die zeilenbasiert gelieferte Daten in die benötigte Spaltenstruktur umwandelt.\footcite[Vgl.][]{was-hana}\\Ein Problem, das bei In-Memory-Datenbanken besteht, ist die Erfüllung der sogenannten ACID-Kriterien, die von Datenbanken, bzw. Datenbank-Managementsystemen erfüllt werden müssen. Das Akronym \glqq{}ACID\grqq{} steht dabei für die Eigenschaften \underline{A}tomicity (Atomarität), \underline{C}onsistency (Konsistenz), \underline{I}solation (Abgrenzung) und \underline{D}ura-bility (Dauerhaftigkeit). Von Natur aus kann die Anforderung der Dauerhaftigkeit bei der Datenhaltung im Hauptspeicher nicht gegeben werden, da dieser flüchtig ist und im Falle einer Stromunterbrechung durch einen Stromausfall, oder Systemabsturz seine Daten verliert. Um dieses Problem zu lösen, gibt es in HANA einen \glqq{}Persistenz Layer\grqq{}, der dafür sorgt, dass die Datenbank in regelmäßigen Abständen (standardmäßig 300 Sekunden) in Form von \glqq{}Savepoints\grqq{} (Sicherheitspunkten) auf einem Dauerspeicher gesichert wird. Um auch nicht beendete Transaktionen auf der Datenbank wiederherzustellen, gibt es die Möglichkeiten den letzten Zustand anhand von Logs zu rekonstruieren. Auf demselben Weg ist es auch möglich den vorherigen Zustand nach Abbruch einer Datenbanktransaktion wiederherzustellen, wodurch zugleich auch die Eigenschaft der Konsistenz gewährleistet wird.\footcite[Vgl.][]{rz10-acid}

\subsubsection{SAP S/4HANA}
\label{kap:S4HANA} 
SAP S/4HANA ist die neuste Generation des ERP-Produkts von SAP. Mit S/4HANA wurde im Jahr 2015 die vorherige Generation SAP-ERP (R/3) abgelöst und vollstän-dig auf Basis der neuen HANA-Datenbanktechnologie (siehe Kapitel \ref{kap:HANA}) entwickelt und angepasst. Dazu kommt, dass mit S/4HANA eine neue Benutzeroberfläche für Webanwendungen eingeführt wurde, den sogenannten FIORI-Anwendungen.\footcite[Vgl.][]{was-hana} Der Name \glqq{}S/4HANA\grqq{} setzt sich dabei aus dem \glqq{}S\grqq{} für \glqq{}Suite\grqq{}, also der Anwendungssuite von SAP, der \glqq{}4\grqq{} für die vierte Produktgeneration des SAP-ERP-Systems und \glqq{}HANA\grqq{} für die bereits erwähnte Datenbanktechnologie zusammen.\footcite[Vgl.][]{rz10-s4hana}\\SAP verspricht mit S/4HANA viele Neuerungen und Vereinfachungen im System, die in sogenannten \glqq{}Simplification-Lists\grqq{} (Vereinfachungslisten) beschrieben werden. Diese Vereinfachung und Verschlankung des Systems geschieht beispielsweise durch das Zusammenlegen von bestehenden Funktionen, wie dem Verschmelzen der Kreditoren und Debitoren zu Geschäftspartnern, und durch die Reduzierung des Datenmodells, indem nicht mehr benötigte Komponenten, wie z.B. durch HANA überflüssig gewordene Aggregationstabellen, entfernt werden. Dies hat jedoch den Nachteil, dass der Umstieg auf S/4HANA viele Hürden mit sich bringt, da vieles bei einer Datenmigration nicht eins zu eins übernommen werden kann.\footcite[Vgl.][]{ibsolution}\\Eine weitere Neuerung von S/4HANA ist die Vermarktung entweder als Cloud-Lösung oder, wie bisher, als On-Premise-Lösung, also als klassische Installation auf einem vom Unternehmen gestellten Server. Die Cloud-Lösung bietet SAP S/4HANA sowohl als \glqq{}Public Cloud\grqq{}-Lösung auch als \glqq{}Private Cloud\grqq{}-Lösung an.
\begin{figure}[h]
    \centering
    \includegraphics[scale=1.3]{./Bilder/HANA-Varianten.png}
    \caption[S/4HANA Lösungen]{Angebotene Lösungsvarianten von S/4HANA (Quelle: \cite[][]{rz10-s4hana})}
\end{figure}
\\Mit der Public-Cloud bietet die SAP ihre ERP-Lösung erstmalig als Software-as-Service (SaaS)-Produkt an, was zur Folge hat, dass der Kunde sich weder um die Bereitstellung der benötigten Hardware, die Installation noch der Aktualisierung der Software kümmern muss, da dieses zentral durch die SAP auf den SAP-eigenen Servern geschieht. Dies hat zum einen für die SAP den Vorteil, dass sie schnell Fehler beheben und Aktualisierungen ausrollen können, die direkt für die Kunden verfügbar sind, ohne, dass diese erst einen Patch oder ein Funktion-Update einspielen müssen. Auch lässt sich die Lizenzierung des Produkts vereinfachen, indem der Kunde nun per Benutzerzugang zahlt und nicht mehr für eine Softwarelizenz. Für die Kunden bringt eine solche Cloud-Lösung ebenfalls Vorteile mit sich, da, wie bereits oben erwähnt, keine eigene Hardware mehr vorgehalten und gewartet werden muss. Dazu kommt, dass die erstmalige Inbetriebnahme des Systems vereinfacht wird, da auch keine Serveraufsetzung und Installation der Software erfolgt, sondern die Zugänge in der Regel online gebucht und anschließend direkt genutzt werden können. Außerdem ist mit einer solchen Lösung die IT-Sicherheit des ERP-Systems bestärkt, da der Kunde sich nun nicht mehr selbst um Backups, Systemupdates oder Rückfallebenen kümmern braucht, sondern die Verantwortung auf den Serviceprovider, also die SAP, umgewälzt wird.\footcite[Vgl.][]{saas} Die SAP-Public Cloud hat jedoch den Nachteil, dass sich viele unternehmensspezifische Anpassungen und Eigenentwicklungen, die als \glqq{}Customizing\grqq{} in der Branche weitverbreitet sind, nicht durchführen lassen, da die Public Cloud als reine Standardlösung auftritt. Die SAP richtet sich mit dem Produkt dadurch vor allem an kleine und mittelständische Unternehmen, die zum einen durch den geringen Aufwand profitieren, auf der anderen Seite aber auch keine Einbußen durch fehlende Eigenentwicklungen und Anpassungsmöglichkeiten haben.\footcite[Vgl.][]{rz10-s4hana}\\Die Private-Cloud-Lösung von S/4HANA (als \glqq{}HANA Enterprise Cloud (HEC)\grqq{} von SAP vermarktet), bringt die Vorteile der Public Cloud mit sich, ermöglicht aber auch Customizing und Eigenentwicklungen. Dies geschieht, indem der Kunde, mit dem Produkt der Private-Cloud, einen eigenen Server bei der SAP anmietet, sich aber dennoch nicht um die Wartung, Aktualisierung, etc. kümmern muss, da dies wie bei der Public-Cloud durch den Serviceprovider geschieht. Die SAP richtet sich mit dem Angebot vor allem an Unternehmen, die bereits SAP eingeführt haben und bereit sind in die Cloud zu wechseln, aber nicht auf ihre eigenen Systemanpassungen, Abweichungen vom Standard, verzichten möchten, bzw. können.\footcite[Vgl.][]{rz10-hana}\glqq{}Die HEC ist eine End-to-End-Lösung, die eine umfassende Cloud-Infrastruktur und Managed Services für In-Memory-Anwendungen, Datenbanken und Plattformen bietet.\grqq{}\footcite[Vgl.][]{rz10-hana}\\Die dritte von SAP angebotene S/4HANA-Variante ist die On-Premise Lösung (Vorort-Lösung), die der Lösungsvariante der vorherigen SAP Generation entspricht. Hierbei wird der Server durch den Kunden selbst gestellt und dieser ist somit selbst für die Installation, Wartung und Sicherheit verantwortlich.\footcite[Vgl.][]{rz10-s4hana} Das ermöglichtet dem Kunden eine größtmögliche Flexibilität, birgt aber Risiken durch etwaige Sicherheitslücken, Systemausfällen oder Datenverlust. Auch ist im Vergleich zu den vorherigen Generationen die Anforderungen an die benötigte Server-Hardware gestiegen, da durch die neue HANA-Datenbank mehr Arbeitsspeicher benötigt wird (siehe Kapitel \ref{kap:HANA}). Die SAP richtet sich mit der On-Premise-Lösung vor allem an größere Unternehmen und Konzerne, die bereits SAP in der Benutzung haben und keine Änderungen an ihrer Infrastruktur und den Prozessen zur Systempflege und Wartung vornehmen möchten.  

\subsection{Transformation}
\subsubsection{Definition}
Unter einer Transformation versteht man im allgemeinen einen grundlegenden Wandel, der durch bestimmte Faktoren, wie z.B. einer sprunghaft wirtschaftlichen, oder technologischen Entwicklung hervorgerufen wird. Die Transformation hält dabei in der Regel über einen längeren Zeitraum an und ist erst beendet, sobald sich die neu geschaffenen Strukturen etabliert und gefestigt haben.\footcite[Vgl.][]{difu}\\ Im betriebswirtschaftlichen Kontext versteht man unter einer Transformation (oder auch Business Transformation) die gezielte Umgestaltung eines Unternehmens und seiner Geschäftsprozesse, um auf veränderte Bedingungen am Markt einzugehen und sich ihnen anzupassen. Dabei ist das Ziel durch effizientere und vereinfachte Geschäftsprozesse einen Mehrwert in Form von niedrigeren Kosten bei gleichbleibender, oder bestenfalls verbesserter Qualität zu erreichen und dabei zusätzlich die Kundenzufriedenheit zu steigern.\footcite[Vgl.][]{leanix}

\subsubsection{Die vier R der Transformation}
In den 1990er-Jahren wurde durch Gouillart und Kelly das Modell der \glqq{}Vier R der Transformation\grqq{} \footcite[Vgl.][]{4r-modell} entwickelt, was eine mögliche Form der Business Transformation darstellen soll. Aus diesem Modell hat die Beratungsgesellschaft Gemini Consulting (später in der Capgemini SE aufgegangen)\footcite[Vgl.][]{gemini-died} ein Produkt entwickelt, indem die vier R für vier verschiedene Transformationsdimensionen stehen:\\
\begin{figure}[h]
    \centering
    \includegraphics[scale=0.5]{Bilder/businesstransformationManagementportal.png}
    \caption[Die vier R der Transformation]{Die vier R der Transformation (Quelle: \cite[][]{4r-modell})}
\end{figure}
\begin{itemize}
    \item[] \emph{Reframing (dt. Einstellungsveränderung):} soll in einem Unternehmen dazu beitragen die Sichtweise auf sich selbst zu überdenken um sich dadurch von alten Denkmustern zu befreien. Um diese Einstellungsveränderung anzustoßen ist es wichtig, dass die Mitarbeiter motiviert werden und davon überzeugt sind durch die eingesetze Energie einen Mehrwert zu generieren. Im nächsten Schritt muss anschließend eine Vision definiert werden, die sich erheblich von der präsenten Realität absetzt um im Anschluss daraus Ziele und Messgrößen zu entwickeln. 
    \item[] Mit der \emph{Restructuring (dt. Restrukturierung)} oder auch Umstrukturierung soll der Aufbau eines Unternehmens überdacht werden, mit dem Ziel durch die Maßnahme eine Verschlankung zu erreichen und dadurch effizienter zu werden. Dies kann durch das Zusammenlegen von Abteilungen, aber auch durch Entlassungen erreicht werden. Im Anschluss ist es notwendig, ebenfalls die Geschäftsprozesse an die neue Struktur anzupassen.   
    \item[] \emph{Revitalising (dt. Wiederbelebung):} Mit der Revitalisierung soll Wachstum erzielt werden, indem sich genauer auf die Kunden ausgerichtet wird und die Erfüllung der Kundenbedürfnisse weiter in Mittelpunkt legt. Eine weitere Option für das Erreichen einer Revitallisierung ist die Erprobung neuer Geschäftsfelder um sich dadurch neue Kunden zu erschließen und Wachstum zu generieren.
    \item[] \emph{Renewing (dt. Erneuerung):} zielt schließlich auf die Mitarbeiter des Unternehmen ab, die sich zum einen mit der neuen Situation vertraut machen sollen, zum anderen aber auch neue, nun benötigte, Fähigkeiten zu erlernen, um das Unternehmen in seiner neuen Situation voran zu bringen. Dazu ist es wichtig, Reize für die Mitarbeiter zu schaffen, diesen Weg zu gehen.
\end{itemize}

\subsection{SAP S/4HANA Transformation}
\label{kap:s4hanatrans}
Die S/4HANA Transformation stellt eine besondere Art der Transformation dar, da sie einen Umstieg auf die neuste SAP Generation in Kombination mit einer digitalen Transformation ermöglicht. Diese Möglichkeit besteht, da die vielen Neuentwicklungen und Vereinfachungen in S/4HANA viel Aufwand für die Aktualisierung des Systems bedeuten, aber gleichzeitig ein großes Potential an Möglichkeiten zur Optimierung der abgebildeten Prozesse ermöglichen. Dadurch wird ein umfassender, innerbetrieblicher Wandel, hin, zu einem digitalisierten Unternehmen mit intelligenten Geschäftsprozessen ermöglicht, unterstützt durch die Echtzeit-Auswertungen und den Möglichkeiten der Business Intelligence, die S/4HANA bietet.
\\Auf Grund der langen Laufzeit von SAP-ERP, bzw. R/3, das seit Beginn der 1990er-Jahre verfügbar war, gibt es viele Systeme, die bereits sehr lange im Einsatz und somit sich historisch gewachsen sind. Während dieser Entwicklung wurden die Systeme immer weiter an die Erfordernisse des jeweiligen Unternehmens angepasst, sodass teils ein großes Delta zum Industriestandard entstanden ist. Mit einer S/4HANA-Transformation ist es nun möglich, alle Geschäftsprozesse zu überdenken und zu optimieren, sowie weitere Geschäftsprozesse zu digitalisieren, da über die Jahre der Funktionsumfang, und somit die Möglichkeit der Abbildung von Geschäftsprozessen, von SAP-ERP stark angewachsen ist und somit, im Vergleich zum Zeitpunkt der Einführung der meisten Systeme, sich viel mehr Möglichkeiten ergeben, Prozesse zu digitalisieren.\footcite[Vgl.][]{s4-interview}
\\Dazu ist es wichtig, sich im Vorfeld mit der eigenen Unternehmensvision und -strategie auseianderzusetzen, und auf Basis derer, Entscheidungen über die angestrebte Prozesslandschaft und das Betriebsmodell zu treffen. Dadurch ist man in der Lage, den möglichst besten Transformationspfad zu beschreiten und die Lösung für seine Probleme, zu einem bestmöglichen Kosten/Nutzen-Verhältnis, zu erzielen.\footcite[Vgl.][]{ao-blog}
Zur Durchführung einer S/4HANA-Transformation gibt es unterschiedliche Ansätze, die sich über die letzten Jahre etabliert haben. Zum einen ist das der Greenfield-Ansatz (Aufbau auf der \glqq{}Grünen Wiese\grqq{}), der Brownfield-Ansatz (Aufbau auf dem Altsystem) und verschiedene Hybride Ansätze, die einen Mittelweg zwischen Greenfield und Brownfield darstellen.\footcite[Vgl.][]{ao-blog} Diese Ansätze werden in den folgenden Unterkapiteln genauer behandelt.
%Industrie 4.0
\subsubsection{Greenfield Ansatz}
Der Greenfield-Ansatz sieht einen kompletten Neuaufbau des SAP ERP-Systems vor und baut nicht auf dem R/3-System auf, sodass dieses bis zur Fertigstellung des neuen Systems in Betrieb bleibt. In der neuen S/4HANA-Installation wird währendessen ein komplett neues System geschaffen, in dem alle abzubildenen Geschäftsprozesse komplett neu gedacht werden und das Geschäftsmodell des Unternehmens angepasst wird.\footcite[Vgl.][]{ao-blog} Dabei wird das Ziel verfolgt, möglichst die \glqq{}Best Practises\grqq{} des Industriestandards zu implementieren und auf Eigenentwicklungen zu verzichten, um im Nachhinein den Wartungsaufwand so gering wie möglich zu halten. Eigenentwicklungen und unternehmenseigene Workarrounds sind in der SAP-Branche weit verbreitet, haben jedoch den Nachteil, dass sie hohe Wartungskosten und einen hohen Pflegebedarf mit sich bringen und auch nicht besonders effizient im Umgang mit der HANA-Datenbank sind und dadurch oftmals auch nicht von den S/4HANA-Verbesserungen profitieren können.\\Im Zuge einer Greenfield-Transformation wird auch das Umfeld des SAP-Systems neu bewertet und Umsysteme bei schlechtem Kosten/Nutzen-Verhältnis abgelöst. Dabei wird auch untersucht, ob die Systeme inzwischen redundant sind und durch den SAP-Standard abgelöst werden können. Die Greenfield-Transformation stellt einer der wenigen Chancen dar, das komplette ERP-System neuaufzubauen und die darin enthaltenen Prozesse neu zu denken, da die moderne Softwareentwicklung sich immer weiter von großen, gebündelten Patches entfernt und mehr in die Richtung von kontinuierlichen, inkrementellen Updates geht und somit stetig der Funktionsumfang wächst.\footcite[Vgl.][]{gambit-transformation} Dies ist vorallem in den S/4HANA-Cloud-Lösungen der Fall, da hier die Systeme und die Software durch die SAP betrieben und gewartet werden und somit Aktualisierungen noch einfacher durchgeführt werden können (siehe Kapitel \ref{kap:S4HANA}). Im Vergleich zum Brownfield-Ansatz ist der Aufwand einer Greenfield-Transformation sehr hoch, da die Implementierung der neuen Prozesse viel Zeit benötigt und die neuen Prozesse auch nach der Einführung getestet werden müssen, ob sie die Anforderungen des Unternehmens abdecken. Auch ist es essentiell, dass die Benutzer in das neue System und die neuen Prozesse eingeführt werden, es hier zu gravierenden Unterschieden in den Arbeitsabläufen kommen kann.\footcite[Vgl.][]{gambit-transformation} 

\subsubsection{Brownfield Ansatz}
Der Brownfield-Ansatz basiert auf der Überlegung auf dem bestehenden SAP-ERP-System der dritten Generation aufzubauen und es nach S/4HANA zu konvertieren. Dazu werden die bestehenden und bereits abgebildeten Prozesse nach S/4HANA übertragen und nicht, wie im Greenfield-Ansatz, komplett neuaufgebaut. Dies bietet für die Unternehmen eine schnellere Implementierung, da die bestehenden Strukturen nur angepasst werden müssen und historischen Daten und Eigenentwicklungen erhalten bleiben. Auch bietet dieser Ansatz Vorteile für die Anwender des Systems, da diese sich nicht auf neue Prozesse einstellen müssen, sondern ihre verinnerlichten Arbeitsabläufe weiterhin anwenden können. Dadurch kann nach Abschluss der Transformation mit der selben Effizienz weitergearbeitet werden und es kommt zu weniger Fehlern.\footcite[Vgl.][]{gambit-transformation}
\\Die Nachteile des Brownfield-Ansatzes sind jedoch, dass die historisch gewachsene Komplexität des Systems mit all seinen Eigenentwicklungen und Workarrounds erhalten bleibt und das volle Potential von SAP S/4HANA nicht genutzt werden kann. Auch bleibt dadurch eine Annäherung an die Best-Practises des Industriestandards aus, da eine Bewertung der Prozesse, der Eigenentwicklungen und der Umsysteme nicht stattfindet, was eine Gleichbleibung, oder im schlimmsten Fall eine Steigerung der Betriebs- und Wartungskosten bedeutet, da die Möglichkeit bestehen, dass diese nicht gut mit S/4HANA harmonieren. Außerdem ist die Nutzung der Public-Cloud-Lösung mit dem Brownfield-Ansatz in der Regel nicht möglich, da diese keine Möglichkeit für Eigenentwicklungen und tiefgreifende Anpassungen bietet, sondern nur die Standardprozesse abbildet. Aus diesen Gründen richtet sich der Brownfield-Ansatz vor allem an Unternehmen, die ihr SAP-ERP noch nicht lange in Betrieb haben, nah am Industriestandard ihrer Prozesse arbeiten und dadurch kaum Eigenentwicklungen und historische \glqq{}Altlasten\grqq{} haben.\footcite[Vgl.][]{gambit-transformation}

\subsubsection{Hybride Ansätze}
Als dritte Möglichkeit bieten viele Beratungsunternehmen einen eigenen, hybriden Ansatz an (teilweise auch \glqq{}Bluefield\grqq{} genannt), der die Vorzüge einer Greenfield-Transformation mit denen einer Brownfield-Transformation vereinen soll. Diese Ansätze unterscheiden sich von Beratungsunternehmen zu Beratungsunternehmen, basieren jedoch zu meist auf dem Neuaufsetzen eines S/4HANA-Systems mit einer anschließenden Migration der benötigten Daten. In diesem System werden dann die S/4HANA-bezogenen Anpassungen vorgenommen. Die bestehenden Prozesse und Entwicklungen im Altsystem werden bewertet und mit ins neue System übernommen, wenn sie essentiell sind und nicht durch den SAP-Standard abgelöst werden können. Diese hybriden Ansätze haben den Vorteil, dass historische Daten übernommen werden können und flexibel entschieden werden kann, welche bestehenden Strukturen übernommen werden. Somit kann dies ganz nach Wunsch der Kunden geschehen und es kann dennoch, wenn auch nur in Teilen, von den Vorzügen S/4HANAs profitiert werden.\footcite[Vgl.][]{hybrideransatz}\\
Im späteren Verlauf dieser Arbeit wird noch einmal genauer auf das Vorgehensmodell zu einem hybriden Ansatz der \glqq{}adesso orange AG\grqq{} eingegangen.


%Grundlagen
\section{Unternehmerischer Kontext}
\subsection{Die adesso orange AG}
\subsubsection{Vorstellung des Unternehmens}

\subsubsection{Geschäftsmodell}

\subsection{Vorgehensmodell S/4HANA Transformation}
\subsubsection{Aufbau}

\subsubsection{Phasen}

\subsubsection{Tools}

\subsubsection{Methodiken}

\subsubsection{Einordnung des BTT}

%Methodik
\section{Methodik}
In diesem Kapitel wird das Vorgehen zu einer Neukonzeptionierung des Business Transformation Trackers erklärt.

\subsection{Untersuchung des Ist-Zustandes}
Zu Beginn findet die Vorstellung des Problems statt, die der Business Transformation Tracker lösen soll und in der erklärt wird, welchen Nutzen und Funktionen er dazu bieten soll. Danach erfolgt die Beschreibung des Ist-Zustandes, in der die aktuelle Implementierung analysiert und der Aufbau beschrieben wird. Im Anschluss findet eine Bewertung der aktuellen Implementierung statt, in der ihre Probleme aufgezeigt und analysiert wird, welche Vor- und Nachteile sie hat. Am Ende wird eine Empfehlung gegeben, wie eine sinnvolle Neuentwicklung aussehen sollte.

\subsection{Requirements Engineering}
Nach der Analyse des Ist-Zustandes folgt die Anforderungserhebung an das neue System. Dazu werden zuerst die einzelnen Stakeholder des BTT analysiert und vorgestellt. Es wird eine Risikoermittlung durchgeführt, in der dargestellt wird, welche potentiellen Auswirkungen auf das Vorhaben möglich sind. Danach wird das Vorgehen zur Anforderungserhebung beschrieben. Diese geschieht durch persönliche Gespräche mit dem Auftraggeber und durch eine Onlinebefragung bei adesso orange, in der die Mitarbeiter nach ihrer Meinung zur aktuellen Implementierung und zu Verbesserungsvorschlägen und Wünschen befragt werden. Im Anschluss erfolgt die Auswertung der Umfrage, in der das aktuelle Meinungsbild beschrieben wird und die Forderungen und Wünsche zusammengefasst dargestellt werden. Schließlich folgt die Spezifikation der Anforderungen, die sich an einer von Helmut Balzert, in seinem Buch \glqq{}Lehrbuch der Softwaretechnik: Basiskonzepte und Requirements Engineering\grqq{}, vorgestellten Anforderungsschablone orientiert. Jedoch wird dabei kein seprates Lasten- und Pflichtenheft erstellt, sondern diese in einer \glqq{}Requirements Specification\grqq{} zusammengefasst. Die Spezifikation der funktionalen Anforderungen erfolgt außerdem zusammengefasst in Anwendungsfällen, die die Arbeitsvorgänge im System darstellen. Diese Anwendungsfälle werden ebenfalls in Aktivitätsdiagrammen dargestellt, um ihre Abläufe zu visualisieren. 

\subsection{Entwicklung der fachlichen Lösung}
Nach der Anforderungsspezifikation erfolgt eine Datenmodellierung, indem aus den Anwendungsfällen Klassen ermittelt und diese im Anschluss durch Attribute, Multiplizitäten und Operation weiter spezifiziert werden. Dazu wird ein Übersichtsklassen-diagramm und ein erweitertes Klassendiagramm erstellt, sowie ein Paketdiagramm angefertigt, das die Klassen zu Paketen zusammenfasst.

\subsection{Prototypentwicklung}
Nachdem die Konzeption und die Datenmodellierung abgeschlossen sind, erfolgt die Entwicklung eines Oberflächen-Prototyps, der die grafische Benutzeroberfläche des Business Transformation Tracker darstellen. In diesem Prototyp werden die definierten Klassen und Beziehungen abgebildet sowie die vorgestellten Anwendungsfälle visualisiert. Dies dient der Vorlage einer späteren Umsetzung und Implementierung des hier vorgestellten Konzeptes. 

%Ist-Zustand
\section{Erhebung des Ist-Zustand}

\subsection{Was bietet das Tool bereits heute}
Zum jetzigen Zeitpunkt exisitiert der Business-Transformation-Tracker bereits in Form eines Excel-Spreadsheets. Dieses wird bereits in einigen Transformationsprojekten des betrachteten Unternehmens verwendet und unterstützt dadurch schon heute die Mitarbeiter in den Projekten. \\Das Tool dient dazu in einem S/4HANA Transformationsprojekt 

Voruntersuchung
•Finden und Definieren des Problems
•Ist-Analyse
•Durchführbarkeits-Untersuchung
-technisch, wirtschaftlich

Aufgaben
•Ist-Situation analysieren
•Hauptanforderungen zusammenstellen
•Lösungsvarianten betrachten
•Empfehlung aussprechen
•Projektkalkulation erstellen
•Projektplan vorschlagen

Ergebnisdokumente
•Lastenheft und Glossar
•Projektkalkulation und Projektplan

Lastenheft (themensammlung)
Erste schriftliche Abstimmung zwischen Auftraggeber und Auftragnehmer über fachliche Basisanforderungen.
Notwendig, da Aufträge typischerweise unvollständig und widersprüchlich sowie unterschiedlich interpretierbar sind.
Wird später zum Pflichtenheft erweitert (daher auch: „grobes Pflichtenheft“)
Im Englischen „Requirements Specification“ (keine Unterscheidung von Lasten-/Pflichtenheft)

\subsection{Welche Verbesserungspotenziale gibt es}

\subsection{Warum verbessern?}

\subsection{Geplante Erweiterungen des Funktionsumfangs}

\subsection{Interviews mit Stakeholdern}

%Anforderungsermittlung
\section{Ermittlung der Anforderungen}
\subsection{Ermittlung der Stakeholder}
Bevor mit der Ermittlung der Anforderungen begonnen wird, müssen zuerst die Stakeholder identifiziert werden, die ein allgemeines Interesse an der zu entwicklenden Software, bzw. dem Tool haben. Dabei handelt es sich um Personen, die entweder von dem fertigen Produkt profitieren, oder die später das Produkt zu ihrer täglichen Arbeit einsetzen und daher ein Interesse am Funktionsumfang und der Benutzerfreundlichkeit haben. 

\subsubsection{Analyse der Stakeholder}
Im Falle des Business-Transformation-Trackers wurde folgende Stakeholder ermittelt:
\begin{itemize}
    \item[] \emph{Oberes Management:} Das obere Management der adesso orange AG ist der Auftraggeber für das Entwicklungsprojekt und hat daher besonderes Interesse in der erfolgreichen Fertigstellung des Projekts und der Produktivsetzung des Systems, um mit dieser Wertschöpfung zu generieren. Dazu kommt, dass das Ziel der Entwicklung die Unterstüzung der Mitarbeiter in den Projekten ist und sich durch den Einsatz eine Effizienzsteigerung und Qualitätsverbesserung erhofft wird. Dadurch besteht die Möglichkeit der Reputationssteigerung gegenüber potentiellen Kunden und somit einer gesteigerten Nachfrage im Vertrieb, was ebenfalls im besonderen Interesse des Managements liegt. In Persona tritt das obere Management im Entwicklungsprojekt als Bereichsleiter \glqq{}SAP Consulting and Development\grqq{} in Erscheinung.
    \item[] \emph{Mittleres Management:} Die Mitarbeiter der adesso orange AG im mittleren Management fungieren in der Regel in der Rolle eines Abteilungs- oder Projektleiters und haben daher ein besonderes Interesse an dem Funktionsumfang an der zu entwicklenen Software, da sie durch den Funktionsumfang direkt in den Projekten profitieren können. So profitieren sie beispielsweise von einer übersichtlichen Ansicht des gesamten Projekts und können durch Auswertungen besser das Projekt verwalten. Außerdem besitzen die Mitarbeiter des mittleren Managements ein Interesse darin, dass das Tool durch die Mitarbeiter verwendet wird, damit die darin geführten Daten stets auf dem aktuellen Stand sind. 
    \item[] \emph{Senior Consultants:} Senior Consultants sind erfahrene Mitarbeiter von adesso orange und arbeiten in der Regel als Projektleiter in kleineren Projekten oder als Teilprojektleiter in Projekten mit größerem Umfang. Sie haben ein großes Interesse in den Funktionsumfang des BTT und sind auch sehr an der Übersichtlichkeit und Benutzerfreundlichkeit der grafischen Oberfläche interessiert, da sie, zusammen mit den Consultants, am intensivsten mit dem Programm arbeiten werden. Dabei stehen die Ziele der Datenkonsistenz und der generellen Verfügbarkeit des BTT im Vordergrund, damit eine reibungslose Arbeit ermöglicht wird.
    \item[] \emph{Consultants:} Die Consultants, bzw. Berater bilden den Kern der Mitarbeiterschaft des auftraggebenen Unternehmens und treten in der Regel als Projektmitarbeiter in Erscheinung. Sie bilden die größte Zielgruppe, da sie am häufigsten mit dem Programm arbeiten werden und dort den Großteil der Datenerfassung durchführen werden. Deshalb ist es vom besonderen Interesse, den Projektmitarbeitern die Arbeit mit dem Produkt möglichst einfach zu machen und besonders auf die Benutzerfreundlichkeit in der Entwicklung zu achten. Dazu kommt, das es wichtig ist, diese Stakeholdergruppe möglichst in die Entwicklung mit einzubeziehen, um Verbesserungsvorschläge und Ideen in die Anforderungen mit aufzunehmen. 
    \item[] \emph{Entwickler:} Die (SAP-)Entwickler des Auftraggebers spielen nur eine untergeordnete Rolle im Kontext des Business Transformation Tracker, da die Befüllung und Auswertung nicht in ihr Aufgabenfeld gehört. Denkbar sind dennoch Szenarien, in denen sie aufgefordert werden einzelne Einträge in dem Programm vorzunehmen, zu denen ihre Expertise benötigt wird. Auch besteht ein großes Interesse an Benutzerfreundlichkeit und Übersichtlichkeit, damit auch bei seltener Nutzung der Umgang mit dem Programm nicht schwer fällt.
    \item[] \emph{Kunden:} Weitere Stakeholder sind die Kunden von adesso orange, da diese ebenfalls Berührungspunkte mit dem Programm haben werden, wenn es Teil ihres Transformationsprojekts wird. Sie haben ein besonderes Interesse an der gesteigerten Effizienz und der gesteigerten Qualität ihrer Transformation, da dies für sie eingesparte Ressourcen in Form von weniger Projekttagen, weniger Aufwänden für die Transformation und geringere Wartungskosten im Anschluss durch die gesteigerte Qualität der Prozesse bedeutet. Dazu kommt, dass es dazu kommen kann, dass in größeren Projekten die Mitarbeiter des Kunden ebenfalls in direkten Kontakt mit dem BTT kommen, um bspw. bei der Erfassung der Prozesse zu unterstützen. Dadurch entsteht ein großes Interesse an der Übersichtlichkeit und der Benutzerfreundlichkeit, damit auch bei einmaliger Benutzung das gewünschte Resultat zustande kommt.
\end{itemize}

\subsubsection{Riskobewertung der Stakeholder}
Von den unterschiedlichen Stakeholdern gehen unterschiedliche Risiken im Bezug auf den Erfolg des Produkts aus. So gibt es auf der einen Seite ein unterschiedliches Konfliktpotential, das durch die unterschiedliche Mächtigkeit der Stakeholder, verschiedene Probleme im späteren Einsatz des Tools herbeiführen kann. So wäre es zum Beispiel denkbar, dass ein Consultant die Arbeit mit dem BTT verweigert, wenn seine persönlichen Anforderungen, in Form von Benutzerfreundlichkeit, außer Acht gelassen werden, oder, das ein Mitarbeiter des mittleren Managements, in Form eines Projektleiters, den Einsatz von Anfang an garnicht erst vorsieht, wenn er der Meinung ist, dass das Tool keinen Mehrwert bietet, wenn seine Anforderungen an das Programm, beispielsweise in Form von Verfügbarkeit und Zuverlässigkeit, nur unzulänglich erfüllt werden.
\begin{figure}[ht]
    \centering
    \includegraphics[scale=0.67]{Bilder/stakeholderRisiko.png}
    \caption[Risikobewertung der Stakeholder]{Subjektive Risikobewertung der genannten Stakeholder}
\end{figure}
Dieses Konfliktpotential lässt sich umgehen, indem die Personengruppen frühzeit in die Anforderungsermittlung mit eingebunden werden und sie dadurch, im Rahmen der Möglichkeiten, selbst bei der Produktentwicklung mitwirken können. Besonders bei Stakeholdern mit großer Macht und hohem Konfliktpotential ist es daher wichtig Maßnahmen zu definieren, wodurch dieses gesenkt werden kann.\footcite[Vgl.][S. 504 f.]{balzert}



\subsection{Erhebung der Anforderungen}
\subsubsection{Informationen durch den Auftraggeber}
Zur Ermittlung der Anforderungen fanden mehrere Gespräche mit dem Auftraggeber statt, in denen zum einen auf den aktuellen Ist-Zustand eingegangen wurde, aber auch Ideen und Umsetzungsvorschläge besprochen wurde. Diese Informationen wurden bereits in dem vorangegangenen Kapitel 5 in der Problemstellung und in der Beschreibung des Ist-Zustandes untergebracht und werden nun genutzt um daraus die Anforderungen an das in Auftrag gegebene Programm zu entwickeln. Während des Entwicklungsprozess besteht ein enger Kontakt zu dem Auftraggeber, wodurch auftretene Rückfragen schnell beantwortet werden können. 

\subsubsection{Befragung im Unternehmen}
Um von möglichst vielen Stakeholdern Anforderungen an eine Neuentwicklung des Business Transformation Trackers zu erhalten, wurde mit den Mitarbeitern des auftraggebenen Unternehmen, die bereits mit der aktuellen Umsetzung des BTT, bzw. seinem Vorgänger, gearbeitet haben, eine Onlinebefragung durchgeführt. Ziel der Befragung war es zum einen das generelle Meinungsbild der Mitarbeiter zu dem BTT zu erfassen und zum anderen mögliche Verbesserungsvorschläge und Ideen der Stakeholder aufzugreifen, um daraus Anfoderungen an einen Neuaufbau des BTT zu entwickeln. Die Umfrage richtet sich dabei an alle internen Stakeholder, das heißt an die Mitarbeiter des oberen und mittleren Mangements, an die Senior Consultants, Consultants und Entwickler. Dadurch soll ein möglichst breites Bild entstehen, dass alle Interessen abdeckt, sodass kein Stakeholder vernachlässigt wird.\\Die Umfrage wurde mit der Online-Plattform \glqq{}Microsoft Teams\grqq{} umgesetzt, das Bestandteil der im Unternehmen eingesetzen Softwaresuite \glqq{}Microsoft 365\grqq{} ist. Die Umfrage wurde anonym durchgeführt, mit der Möglichkeit am Ende freiwillig seine Kontaktdaten anzugeben, um Rückfragen zu den gegebenen Antworten und Vorschlägen zu ermöglichen. Der Fragenkatalog bestand aus drei Abschnitten, zuerst allgemeine Fragen zur Person und zur Position im Unternehmen, als nächstes mit Fragen zur Meinung über den BTT und zum Schluss mit der Möglichkeit Verbesserungsvorschläge und Ideen anzugeben. Um dem Betriebsklima im Unternehmen gerecht zu werden, wurde in der Umfrage auf die förmliche Anrede der Befragten verzichtet.

\subsubsection{Ergebnisse der Umfrage}



%Anforderungsspezifikation
\section{Spezifikation der Anforderungen}
Im nun folgenden Unterkapitel werden die im letzten Kapitel, durch Onlinebefragung und in persönlichen Gesprächen, ermittelten Anforderungen spezifiziert, das heißt, systematisch ausgewertet. Es wird aufgrund einer nichtvorhandenen Ausschreibung des Projekts und des geringen Projektumfangs auf ein seperates Lasten- und Pflichtenheft verzichtet und stattdessen die Anforderungen in der hier beginnenden \glqq{}Requirements Specification\grqq{}, zu deutsch \glqq{}Anforderungsspezifikation\grqq{}, niedergeschrieben. Dazu wird sich an der von Helmut Balzert beschriebenen \glqq{}Schablone[n] für Lastenheft, Pflichtenheft und Glossar\grqq{}\footcite[S. 492]{balzert} orientiert. In dieser werden zuerst die Visionen und Ziele des Entwicklungsprojekt verfasst, danach die Rahmenbedingungen denen die Entwicklung unterliegt, im Anschluss der technische Kontext, in dem sich die Entwicklung abspielt und dann erst die funktionalen Anforderungen, die die Kernfunktionalität des Systems beschreiben gefolgt von den nichtfunktionalen Anforderungen, bzw. den Qualitätsanforderungen, in denen die messbare Qualität und das Verhalten des Systems beschrieben wird.\footcite[Vgl.][S. 492 ff.]{balzert}. Die Anforderungen sind natursprachlich verfasst und verfügen über einen einzigartigen Identifikator, um im späteren Verlauf auf sie verweisen zu können. Diese sind so aufgebaut, dass \glqq{} [j]ede Anforderung [..] mit einem Buchstaben [beginnt] [...], gefolgt von einer Zahl, eingschlossen in Schrägstriche. Der Anforderungstyp wird durch einen Buchstaben gekennzeichnet [...].\grqq{} \footcite[S. 493]{balzert}

\subsection{Visionen und Ziele}
Die hier aufgezählten Visionen und Ziele sind Ausdruck der mit dem fertigen Produkt zu erreichenden Zukunft. Visionen sind dabei abstrakter und generisch verfasst, Ziele konkretisieren diese dann im Anschluss.\footcite[Vgl.][S. 457]{balzert}
\begin{itemize}
    \item[] \emph{/V10/} Der Auftraggeber soll durch den Business Transformation Tracker eine Qualitätssteigerung und Effizienzverbesserung in seinen Transformationsprojekten erreichen.
    \item[] \emph{/V20/} Die Anwender sollen mit dem Business Transformation Tracker während des gesamten Projektzeitraums die in SAP umgesetzten Prozesse erfassen und nachverfolgen können.
    \item[] \emph{/V30/} In jedem adesso active transformation -Projekt soll der Business Transformation Tracker eingesetzt werden.
    \item[] \emph{/V40/} Das Produkt soll dem Anwender eine angenehme User Experience bieten und muss ihn in seiner Arbeit produktiv unterstützen.\\
\end{itemize}

\begin{itemize} 
    \item[] \emph{/Z10/} Der Business Transformation Tracker soll zu jedem Zeitpunkt den aktuellen Fortschritssgrad ausgeben können, um schnell eine Übersicht zu erhalten.
    \item[] \emph{/Z20/} Dem Anwender soll es möglich sein, unterschiedliche Projekt aufrufen zu können.
    \item[] \emph{/Z30/} Die Ziele der Informationssicherheit (Authentizität, Vertraulichkeit, Integrität) dürfen nicht verletzt werden.
    \item[] \emph{/Z40/} Alle bereits jetzt implementierten Funktionen werden in die Neuentwicklung übernommen.         
    \item[] \emph{/Z50/} Der Business Transformationen Tracker soll den Funktionsumfang der jetzigen Lösung überbieten.  
    \item[] \emph{/Z60/} Das Anlegen eines Projektes im BTT dauert nicht länger als eine Minute.
    \item[] \emph{/Z70/} Die Erstellung eines Prozesschrittes ist dem Benutzer intuitiv möglich.
    \item[] \emph{/Z80/} Die Anwendung ist auf den verbreitetsten Systemen, Windows, Mac und Linux, einsetzbar.
\end{itemize}

\subsection{Rahmenbedingungen}
Als Rahmenbedingungen bezeichnet man Einschränkungen, die in der Entwicklung der Software berücksichtigt werden müssen. Diese sind entweder technischer oder organisatorischer Natur.\footcite[Vgl.][S. 459 f.]{balzert}
\begin{itemize}
    \item[] \emph{/R10/}
    \item[] \emph{/R20/}
\end{itemize}

\subsection{Kontext und Überblick}
Der Kontext beschreibt die technische Umgebebung, in die die Entwicklung eingebettet ist und welche Abhängigkeiten und Schnittstellen zu anderen Systemen exisitieren.\footcite[Vgl.][S. 461 f.]{balzert} 
\begin{itemize}
    \item[] \emph{/K10/}
    \item[] \emph{/K20/}
\end{itemize}

\subsection{Funktionale Anforderungen}
Die Funktionalen Anforderungen beschreiben den Funktionsumfang des Systems. Sie werden im folgenden auf oberster Abstraktionsebene beschrieben und durch Anwendungsfälle (Use-Cases) zusammengefasst mit Hilfe von Sequenzdiagrammen und Anwendungsfalldiagrammen dargestellt.\footcite[Vgl.][S. 496]{balzert}Für die Beschreibung wird auf eine Anwendungsfallschablone zurückgegriffen, die die Eigenschaften des Anwendungsfall systematisch abfragt. Die Eigenschaften sind, das Ziel des Anwendungsfall, die Kategorie, die angibt wie häufig der Anwendungsfall ausgeführt wird, die Vorbedingung, die Nachbedingung bei Erfolg, die Nachbedingung bei Misserfolg, die Akteure des Anwendungsfall, das Auslösende Ereignis, die Beschreibung in einzelnen Schritten, die Erweiterung und mögliche Alternativen.\footcite[Vgl.][S. 261]{balzert}
\begin{figure}[h]
    \centering
    \includegraphics[scale=0.67]{./Bilder/Anwendungsfalldiagramm.png}
    \caption[Anwendungsfalldiagramm]{Anwendungsfalldiagramm mit Akteueren}
    \label{fig:Anwendungsfalldiagramm}
\end{figure}
In Abbildung \ref{fig:Anwendungsfalldiagramm} ist eine Übersicht der aus den Anforderungen erarbeiteten Anwendungsfälle zu sehen, die in den nachfolgenden Unterkapiteln anhand der oben beschriebenen Schablone genauer beschrieben werden.

\subsubsection{Anwendungsfall 1: User verwalten}
\underline{\emph{Ziel:}}\\
Es werden die für einen User hinterlegten Daten geändert. Dies kann zum einen das Passwort sein, aber auch die Stammdaten, die Rolle, oder die Zuordnung zu einem Projekt, oder Teilprojekt.\\
\underline{\emph{Kategorie:}} \\
Sekundär\\
\underline{\emph{Vorbedingung:}} \\
Der Anwender hat sich zuvor im System mit seinem Benutzernamen und Passwort angemeldet.\\
\underline{\emph{Nachbedingung Erfolg:}} \\
Die Daten und Zuordnungen des Benutzer wurden wie gewünscht angepasst.\\
\underline{\emph{Nachbedingung Fehlschlag:}} \\
Die Daten und Zuordnungen des Benutzers bleiben unverändert.\\
\underline{\emph{Akteure:}} \\
Administrator, Projektleiter\\
\underline{\emph{Auslösendes Ereignis:}} \\
Die Daten oder Zuordnungen des Benutzer müssen angepasst werden.\\
\underline{\emph{Beschreibung:}}
\begin{itemize}
    \item [1] Die Benutzerübersicht wird aufgerufen
    \item [2] Der gewünschte Benutzer wird ausgewählt.
    \item [3] Die Stammdaten des Benutzer werden editiert.
    \item [4] Die veränderten Daten werden gesichert.
\end{itemize}
\underline{\emph{Erweiterung:}}
\begin{itemize}
    \item [2a] Der Nutzer ist nicht vorhanden und wird angelegt.
    \item [3a] Der Benutzer wird einem Projekt zugeordnet.
    \item [3b] Der Benutzer wird einem Teilprojekt zugeordnet.
    \item [3c] Der Benutzer wird aus einem Projekt gelöscht.
    \item [3d] Der Benutzer wird aus einem Teilprojekt gelöscht.
    \item [3e] Das Passwort des Benutzer wird zurückgesetzt.
\end{itemize}
\underline{\emph{Alternativen:}}
\begin{itemize}
    \item [2b] Der Benutzer wird gelöscht.
    \item [3f] Die Daten sind korrekt und der Vorgang wird abgebrochen.
\end{itemize}

\subsubsection{Anwendungsfall 2: Projekt anlegen}
\underline{\emph{Ziel:}}\\
Ein neues Projekt wird hinzugefügt und die Stammdaten des Projekts werden erfasst.\\
\underline{\emph{Kategorie:}} \\
Primär\\
\underline{\emph{Vorbedingung:}} \\
Der Anwender hat sich zuvor im System mit seinem Benutzernamen und Passwort angemeldet und das Projekt ist noch nicht angelegt.\\
\underline{\emph{Nachbedingung Erfolg:}} \\
Das Projekt wird angelegt und die Stammdaten des Projekts werden wie gewünscht hinterlegt, die benötigten Projektphasen sind vorhanden und Teilprojekte sind ebenfalls angelegt.\\
\underline{\emph{Nachbedingung Fehlschlag:}} \\
Es wird kein Projekt angelegt.\\
\underline{\emph{Akteure:}} \\
Administrator, Projektleiter\\
\underline{\emph{Auslösendes Ereignis:}} \\
Es wird ein neues Projekt begonnen.\\
\underline{\emph{Beschreibung:}}
\begin{itemize}
    \item [1] Der Anwender ruft die Projektübersicht auf.
    \item [2] Ein neues Projekt wird hinzugefügt.
    \item [3] Die Stammdaten des Projekts werden hinterlegt, in Form eines Bezeichners, Beschreibung, Projektzeitraum, Kunde, und ggf. Bemerkungen.
    \item [4] Es wird ein Teilprojekt hinzugefügt.
    \item [5] Es wird eine Projektphase hinzugefügt.
    \item [6] Es werden die vorgesehenen Mitarbeiter hinzugefügt.
    \item [7] Die veränderten Daten werden gesichert.
\end{itemize}
\underline{\emph{Erweiterung:}} \\
\begin{itemize}
    \item [4a] Es werden weitere Teilprojekte hinzugefügt. 
    \item [5a] Es werden weitere Projektphasen hinzugefügt.
    \item [5b] Die Attribute der Projektphasen, die standardmäßig vorgegeben sind, werden angepasst. 
\end{itemize}
\underline{\emph{Alternativen:}} \\
./.

\subsubsection{Anwendungsfall 3: Fortschritt erfassen}
\underline{\emph{Ziel:}}\\
Der aktuell erarbeitete Fortschritt wird durch die Bearbeitung der Attribute eines Prozessschrittes dokumentiert.\\
\underline{\emph{Kategorie:}} \\
Primär\\
\underline{\emph{Vorbedingung:}} \\
Der Anwender hat sich zuvor im System mit seinem Benutzernamen und Passwort angemeldet. Es ist ein Projekt mit Teilprojekten und Projektphasen angelegt und der Bearbeiter ist dem Projekt zugeordnet.\\
\underline{\emph{Nachbedingung Erfolg:}} \\
Es wurden Projektphasenattribute in den zu bearbeitenen Prozessschritten verändert und die Fortschrittsanzeige verändert sich.\\
\underline{\emph{Nachbedingung Fehlschlag:}} \\
Es werden keine Projektphasenattribute verändert, die Fortschrittsanzeige bleibt gleich.\\
\underline{\emph{Akteure:}} \\
Teilprojektleiter, Projektmitarbeiter\\
\underline{\emph{Auslösendes Ereignis:}} \\
Außerhalb des IT-Systems wurde eine Aufgabe abgearbeitet, die anschließend dokumentiert werden muss.\\
\underline{\emph{Beschreibung:}} 
\begin{itemize}
    \item [1] Es wird die Übersicht des Projekts aufgerufen.
    \item [2] Es wird das dem Benutzer zugeordnete Teilprojekt aufgerufen.
    \item [3] Es wird der zu bearbeitene Prozess ausgewählt und innerhalb des Prozesses der entsprechende Prozessschritt.
    \item [4] Das zu ändernde Projektphasenattribut wird verändert.
    \item [5] Die Änderungen werden gespeichert.
    \item [6] Das System verändert automatisch die Fortschrittsanzeige für den entsprechenden Prozess, bzw. die Phase.
\end{itemize}

\underline{\emph{Erweiterung:}} 
\begin{itemize}
    \item [1a] Wenn der Benutzer mehreren Projekten zugeordnet ist, wechselt er zuvor in das zu bearbeitene Projekt. 
    \item [4a] Es werden weitere Projektphasenattribute verändert.
\end{itemize}
\underline{\emph{Alternativen:}}
\begin{itemize}
    \item [6a] Wenn keine Änderung stattgefunden hat, verändert sich die Fortschrittsanzeige nicht.
\end{itemize}

\subsubsection{Anwendungsfall 4: Fortschritt überprüfen}
\underline{\emph{Ziel:}}\\
Der Fortschritt in den jeweiligen Teilrojekten für die aktuelle Projektphase wird wiedergegeben.\\
\underline{\emph{Kategorie:}} \\
Sekundär\\
\underline{\emph{Vorbedingung:}} \\
Der Anwender hat sich zuvor im System mit seinem Benutzernamen und Passwort angemeldet. Es ist ein Projekt mit Teilprojekten und Projektphasen angelegt und die auswertende Person ist dem Projekt zugeordnet.\\
\underline{\emph{Nachbedingung Erfolg:}} \\
Es wird die gewünschte Auswertung ausgegeben.\\
\underline{\emph{Nachbedingung Fehlschlag:}} \\
Es wird keine Auswertung ausgegeben, wordurch der Fortschritt individuell bei den Projektmitarbeitern abgefragt werden muss.\\
\underline{\emph{Akteure:}} \\
Projektleiter, Teilprojektleiter, Kunde\\
\underline{\emph{Auslösendes Ereignis:}} \\
Während einer Überprüfung der geleisteten Arbeit soll der aktuelle Fortschritt aufgezeigt werden. Dies kann z.B. im Rahmen eines täglichen Jour Fixes geschehen.
\underline{\emph{Beschreibung:}} 
\begin{itemize}
    \item [1] Es wird die Übersicht des Projekts aufgerufen.
    \item [2] Es wird das Dashboard des Projektes aufgerufen. Dort befindet sich auf einem Blick Fortschrittsanzeigen für alle Teilprojekte in der aktuellen Projektphase.
\end{itemize}
\underline{\emph{Erweiterung:}}
\begin{itemize}
    \item [1a] Wenn der Benutzer mehreren Projekten zugeordnet ist, wechselt er zuvor in das zu bearbeitene Projekt.
\end{itemize}
\underline{\emph{Alternativen:}}
\begin{itemize}
    \item [2a] Es werden nach und nach die einzelnen Teilprojekte aufgerufen um dort eine detailliertere Aufschlüsselung des aktuellen Fortschrittes zu erhalten.
\end{itemize}

\subsubsection{Anwendungsfall 5: Teilprojekt verwalten}
\underline{\emph{Ziel:}}\\
Es werden Änderungen in einem Teilprojekt vorgenommen um die Stammdaten zu ändern, um Projektphasenattribute zu bearbeiten oder um Mitarbeiter dem Teilprojekt hinzuzufügen.\\
\underline{\emph{Kategorie:}} \\
Sekundär\\
\underline{\emph{Vorbedingung:}} \\
Es ist ein Projekt mit Teilprojekten und Projektphasen angelegt und die auswertende Person ist dem Projekt und dem zu verwaltenden Teilprojekt zugeordnet. Ein dem Teilprojekt zuzuordnener Mitarbeiter muss bereits dem Projekt zugeordnet sein.\\
\underline{\emph{Nachbedingung Erfolg:}} \\
Die Änderungen werden im System gespeichert um die Informationen im Teilprojekt werden wie gewünscht angepasst.\\
\underline{\emph{Nachbedingung Fehlschlag:}} \\
Die Änderungen werden nicht gespeichert, die Daten bleiben unverändert.\\
\underline{\emph{Akteure:}} \\
Projektleiter, Teilprojektleiter\\
\underline{\emph{Auslösendes Ereignis:}}\\
Die Daten im Teilprojekt müssen angepasst werden, weil z.B. ein neuer Mitarbeiter hinzu kommt.\\
\underline{\emph{Beschreibung:}} 
\begin{itemize}
    \item [1] Es wird die Übersicht des Projekts aufgerufen.
    \item [2] Es wird die Übersicht des Teilprojektes aufgerufen.
    \item [3] Die Stammdaten des Teilprojektes werden editiert.
\end{itemize}
\underline{\emph{Erweiterung:}}\\
./.\\
\underline{\emph{Alternativen:}}\\
\begin{itemize}
    \item [3a] Es wird ein neuer Mitarbeiter dem Teilprojekt zugeordnet.
    \item [3b] Es wird ein Projektphasenattribut hinzugefügt. 
    \item [3c] Es wird ein Projektphasenattribut entfernt.
    \item [3d] Es wird ein Projektphasenattribut verändert.
\end{itemize}

\subsubsection{Anwendungsfall 6: Prozess erfassen}
\underline{\emph{Ziel:}}\\
Die Prozesse des Kunden sind vollständig inklusive ihrer einzelen Prozessschritte im System erfasst.\\
\underline{\emph{Kategorie:}} \\
Primär\\
\underline{\emph{Vorbedingung:}} \\
Es ist ein Projekt mit Teilprojekten und Projektphasen angelegt und die auswertende Person ist dem Projekt und dem zu verwaltenden Teilprojekt zugeordnet.
\underline{\emph{Nachbedingung Erfolg:}} \\
Es sind neue Prozesse mit ihren jeweiligen Schritten im System hinterlegt.\\
\underline{\emph{Nachbedingung Fehlschlag:}} \\
Es werden keine neuen Prozesse im System erfasst.\\
\underline{\emph{Akteure:}} \\
Teilprojektleiter, Projektmitarbeiter, Kunde\\
\underline{\emph{Auslösendes Ereignis:}} \\
In einem Teilprojekt sollen zu Beginn des Projektes die Prozesse in allen Umfängen erfasst werden.\\
\underline{\emph{Beschreibung:}}
\begin{itemize}
    \item [1] Es wird die Übersicht des Projekts aufgerufen.
    \item [2] Es wird die Übersicht des Teilprojektes aufgerufen.
    \item [3] Die Prozessübersicht des Teilprojekts wird aufgerufen.
    \item [4] Es wird ein neuer Prozess angelegt.
    \item [5] Es werden die Informationen zu dem Prozess erfasst.
    \item [6] Es wird ein neuer Prozessschritt angelegt.
    \item [7] Die Änderungen werden gesichert.
\end{itemize}

\underline{\emph{Erweiterung:}}
\begin{itemize}
    \item [6a] Ein Subprozess wird erfasst.
\end{itemize}
\underline{\emph{Alternativen:}} \\
./.\\



\subsection{Qualitätsanforderungen}
Die nichtfunktionalen Anforderungen, bzw. Qualitätsanforderungen spiegeln Eigenschaften wieder, die das gesamte System und somit alle funktionalen Anforderungen betreffen. Die Qualitätsanforderungen werden anhand unterschiedlicher Kriterien kategorisiert, der \textbf{F}unktionalität, der \textbf{Z}uverlässigkeit, der \textbf{B}enutzbarkeit, der \textbf{E}ffizienz, der \textbf{W}artbarkeit und der \textbf{P}ortabilität.\footcite[Vgl.][S. 494 f.]{balzert} Die ermittelten nichtfunktionalen Anforderungen lauten wie folgt:
\begin{itemize}
    \item[] \emph{/Q10/}
    \item[] \emph{/F00/} Die Anwendung benutzt eine grafische Oberfläche.
    \vspace{0.5cm}
    \item[] \emph{/Q20/}
\end{itemize}

\begin{comment}
    Use cases
    1. Tägliches Statusupdates zur Besprechung des Fortschrittes in den Teilprojekten
    2. Anlegen eines Projektes mit seinen jeweiligen Teilprojekten und Projektphasen, Zuordnung der Rollen
    3. Initiales Erfassen eines Prozesses, mit seinen Subprozessen und den Prozessschritten
    4. Pflegen der Felder einer Projektphase 
    5. Abschließen einer Projektphase.

    Akteuere im System
    - Eigentümer des Projekts, Admin, oberes Mgmnt.
    - Projektleiter (n, normalfall 2)
    - Projektcontroller (read only)
    - Teilprojektleiter
    - Projektmitarbeiter

    \begin{itemize}
        \item[] \emph{/F10/} Die Anwendung benutzt eine grafische Oberfläche.
        
        \item[] \emph{/F20/} Es gibt unterschiedliche Benutzerrollen im System.
        
        \item[] \emph{/F30/} Die Benutzerrollen haben unterschiedliche Berechtigungen.
        \item[] \emph{/F31/} Der Benutzer loggt sich mit Benutzername und Passwort ein.
        
        \item[] \emph{/F40/} Es können ein oder mehrere Projekte angelegt und aufgerufen werden.
        
        \item[] \emph{/F50/} Ein Projekt besteht aus mehreren Projektphasen.
        \item[] \emph{/F51/} Die Projektphasen werden auf die Teilprojekte vererbt.
        
        \item[] \emph{/F60/} Innerhalb eines Projektes können ein oder mehrere Teilprojekte erstellt werden.
        
        \item[] \emph{/F70/} Innerhalb eines Projektes werden Prozesse aufgenommen
        \item[] \emph{/F80/} Die Prozesse können einem Teilprojekt zugeordnet werden oder nicht.
        \item[] \emph{/F90/} Innerhalb eines Prozesses können keine oder mehere Subprozesse aufgenommen werden.
        \item[] \emph{/F100/} Ein (Sub-)Prozess besteht aus einem oder mehreren Prozessschritten
        \item[] \emph{/F110/} Ein Prozessschritt muss einem Teilprojekt zugeordnet sein.
        \item[] \emph{/F120/}      
    \end{itemize}
    
\end{comment}
    

%Datenmodellierung, Klassendiagramme, Klassenspezifikation
\section{Modellierung der Daten}
\subsection{Ermittlung der Klassen}
In diesem Kapitel geht es um die objektorientierte Analyse der in dem letzten Kapitel spezifizierten Anforderungen. Ziel ist es mit Hilfe eines Klassendiagramms die Daten des System visuell darzustellen und die Beziehungen der Klassen untereinander zu modellieren. Diese Klasendiagramme sollen einen tatsächlichen Aufbau des Systems darstellen, der in einem späteren, nicht im Rahmen dieser Arbeit angefertigten, Entwurf und Implementierung als Vorlage dienen soll um die tatsächlichen Klassen aufzubauen. Da dieses Klassendiagramm bereits in der Konzeptionsphase des Systems aufgebaut wird, haben diese Modellierungen keinen Anspruch auf Richtigkeit, da aus technischen oder organisatorischen Gründen zu einem späteren Zeitpunkt noch Änderungen erfolgen können. Diese Änderungen können zum Beispiel aufgrund von technischen Restriktionen der Entwicklungsumgebung oder aufgrund von Wünschen des Auftraggebers auftreten.
\\Um die Übersichtlichkeit zu bewahren werden die Klassen und Beziehungen als erstes aufgezählt und beschrieben. Im Anschluss erfolgt die Darstellung der Klassen in einem Übersichtsklassendiagramm und die Beschreibung der untereinander herrschenden Beziehungen. Danach werden die Klassen in einem erweiterten Klassendiagramm dargestellt, in dem auch die Operationen und Attribute der Klassen visualisiert sind. Im Anschluss erfolgt eine exemplarische Spezifikation von Attributen und Operation anhand von drei ausgewählten Klassen. Zum Schluss erfolgt die Zusammenfassung der Klassen in einem Paketdiagramm.    

\newpage

\subsection{Übersichtsklassendiagramm}
\begin{figure}[h!]
    \centering
    \includegraphics[scale=0.6]{./Bilder/Übersichtsklassendiagramm.png}
    \caption[Übersichtsklassendiagramm]{Übersichtsklassendiagramm}
    \label{fig:Übersichtsklassendiagramm}
\end{figure}

\newpage
\subsection{Klassen und Beziehungen}
\subsubsection{Beschreibung der Klassen}
In den nachfolgenden Tabelle werden zuerst die ermittelten Klassen beschrieben, danach die Beziehungen mit Assoziationen und die Beziehungen mit Generalisierungen.

\begin{xltabular}{\textwidth}{|p{0.3\textwidth}|p{0.642\textwidth}|}
    \hline
    \textbf{Klasse} & \textbf{Beschreibung} \\\hline\hline
    Benutzer & Eine natürliche Person, die ein Benutzerkonto im System hat.\\\hline
    Administrator & Eine natürliche Person, die für die Verwaltung des Systems zuständig ist. \\\hline
    Projektleiter & Eine natürliche Person, die ein S/4HANA-Transformationsprojekt leitet. \\\hline
    Teilprojektleiter & Eine natürliche Person, die ein Teilprojekt in einem S/4HANA-Transformationsprojekt leitet. \\\hline
    Projektmitarbeiter & Eine natürliche Person, die Mitarbeiter in einem S/4HANA-Transformationsprojekt ist. \\\hline
    Kunde &  Eine natürliche Person, die Kunde eines S/4HANA-Transformationsprojekt ist.\\\hline
    Projekt & Eine Einheit, die ein S/4HANA-Transformationsprojekt abbildet. \\\hline
    Teilprojekt & Eine Einheit, die ein Teilprojekt in einem S/4HANA-Transformationsprojekt abbildet. \\\hline
    Projektphase & Ein Zeitraum in einem S/4HANA-Transformationsprojekt\\\hline
    Attribut & Eine Eigenschaft einer Projektphase, die während der S/4HANA-Transformation erfüllt wird. \\\hline
    Boolsches Attribut & Ein Attribut, das einen Ja- oder Nein-Wert speichert.\\\hline
    Textattribut & Ein Attribut, das eine Zeichenkette speichert.\\\hline
    Zahlenattribut & Ein Attribut, das eine Ganzzahl oder Gleitkommazahl speichert. \\\hline
    Prozess &  Ein in SAP abgebildetete Folge von Aktivitäten.\\\hline
    Subprozess &  Ein in sich geschlossener Abschnitt eines Prozess.\\\hline
    Prozessschritt & Eine Aktivität eines Prozess.\\\hline
\end{xltabular}
\captionof{table}[Klassenbeschreibungen]{Beschreibung der ermittelten Klassen}

\subsubsection{Beschreibung der Assoziationen}
Assoziationen sind Beziehungen, die zwischen Klassen bestehen und können unterschiedlicher Natur sein. Assoziationen sind bidirektional und funktionieren deswegen in beide Richtungen.\\
\begin{xltabular}{\textwidth}{|p{0.3\textwidth}|p{0.145\textwidth}|p{0.145\textwidth}|p{0.3\textwidth}|}
    \hline
    \multicolumn{4}{|c|}{\textbf{Klassen mit Assoziationen}}\\\hline\hline
    %%%%%%
    Administrator & 1 & 0..* & Projekt\\\hline
    \multicolumn{4}{|p{0.942\textwidth}|}{Ein Administrator kann kein, oder mehrere Projekte im System erstellen, aber ein Projekt kann nur von genau einem Administrator erstellt werden.}\\\hline\hline
    %%%%%%
    %%%%%%
    Benutzer & 1..* & 0..* & Projekt\\\hline
    \multicolumn{4}{|p{0.942\textwidth}|}{Ein Benutzer kann kein, oder mehreren Projekten im System zugeordnet sein und ein Projekt kann keinem oder mehreren Benutzer zugeordnet werden.}\\\hline\hline
    %%%%%%
    %%%%%%
    Administrator & 1..* & 0..* & Benutzer\\\hline
    \multicolumn{4}{|p{0.942\textwidth}|}{Ein Administrator kann ein keinen oder mehrere Benutzer verwalten und ein Benutzer kann von einem oder mehreren Administratoren verwaltet werden.}\\\hline\hline
    %%%%%%
    Projektleiter & 1..* & 1 & Projektphase\\\hline
    \multicolumn{4}{|p{0.942\textwidth}|}{Ein Projektleiter kann eine oder mehrere Projektphasen erstellen aber eine Projektphase kann nur durch genau einen Projektleiter erstellt werden.}\\\hline\hline
    %%%%%%
    %%%%%%
    Teilprojektleiter & 1 & 0..* & Attribut\\\hline
    \multicolumn{4}{|p{0.942\textwidth}|}{Ein Teilprojektleiter kann kein, oder belieb viele Attribute erstellen, ein Attribut wird jedoch nur durch eine Teilprojektleiter erstellt.}\\\hline\hline
    %%%%%%
    Projektleiter & 1..* & 0..* & Projekt\\\hline
    \multicolumn{4}{|p{0.942\textwidth}|}{Ein Projektleiter verwaltet kein oder mehrere Projekte und ein Projekt wird durch mindestens einen Projektleiter verwaltet}\\\hline\hline
    %%%%%%
    Teilprojektleiter & 1..* & 0..* & Teilprojekt\\\hline
    \multicolumn{4}{|p{0.942\textwidth}|}{Ein Teilprojektleiter verwaltet kein oder mehrere Teilprojekte und ein Teilprojekt wird durch mindestens einen Projektleiter verwaltet}\\\hline\hline
    %%%%%%
    Projektmitarbeiter & 1 & 0..* & Attribut\\\hline
    \multicolumn{4}{|p{0.942\textwidth}|}{Ein Projektmitarbeiter kann kein, oder belieb viele Attribute erstellen, ein Attribut wird jedoch nur durch einen Projektmitarbeiter erstellt.}\\\hline\hline
    Prozessschritt & 0..* & 0..* & Attribut\\\hline
    \multicolumn{4}{|p{0.942\textwidth}|}{Ein Prozessschritt prägt kein, oder beliebig viele Attribute aus und ein Attribut wird durch kein oder beliebig vielen Prozessschritten ausgeprägt.}\\\hline\hline
    Subprozess & 0..1 & 1..* & Prozessschritt\\\hline
    \multicolumn{4}{|p{0.942\textwidth}|}{Ein Subprozess beinhaltet einen oder mehrere Prozessschritte und ein Prozessschritt kann zu keinem oder einem Prozessschritt gehören, da Subprozesse nur optional sind.}\\\hline
\end{xltabular}
\captionof{table}[Klassen mit Assoziationen]{Beschreibung der ermittelten Klassen mit Assoziationen}

\newpage
\subsubsection{Beschreibung der Aggregationen}
Aggregationen sind eine besondere Form von Assoziationen, bei denen Objekte einer Klasse einen Teil von einer anderen Klasse darstellen. Dadurch ist es möglich beispielsweise Besitz darzustellen.\\

\begin{xltabular}{\textwidth}{|p{0.3\textwidth}|p{0.145\textwidth}|p{0.145\textwidth}|p{0.3\textwidth}|}
    \hline
    \multicolumn{4}{|c|}{\textbf{Klassen mit Aggregationen}}\\\hline\hline
    Teilprojekt & 1..* & 0..* & Prozess\\\hline
    \multicolumn{4}{|p{0.942\textwidth}|}{Ein Teilprojekt beinhaltet keinen oder mehrere Prozesse, ein Prozess gehört aber immer zu einem oder mehreren Teilprojekten. Ein Teilprojekt ohne Prozesse kann existieren, ein Prozess ohne nicht mindestens ein Teilprojekt jedoch nicht.}\\\hline\hline
    Prozess & 1 & 0..* & Subprozess\\\hline
    \multicolumn{4}{|p{0.942\textwidth}|}{Ein Prozess kann keine oder mehrere Subprozesse beinhalten, ein Subprozess gehört immer zu genau einem Prozess. Ein Prozess ohne Subprozesse kann existieren, ein Subprozess ohne Prozess jedoch nicht.}\\\hline
\end{xltabular}
\captionof{table}[Klassen mit Aggregation]{Beschreibung der ermittelten Klassen mit Aggregationen}

\subsubsection{Beschreibung der Kompositionen}
Kompositionen sind eine besondere Form von Aggregation, bei denen die Objekte der einen Klasse nicht ohne die Objekte der anderen Klasse existieren können.\\

\begin{xltabular}{\textwidth}{|p{0.3\textwidth}|p{0.145\textwidth}|p{0.145\textwidth}|p{0.3\textwidth}|}
    \hline
    \multicolumn{4}{|c|}{\textbf{Klassen mit Kompositionen}}\\\hline\hline
    Projekt & 1 & 1..5 & Projektphase\\\hline
    \multicolumn{4}{|p{0.942\textwidth}|}{Ein Projekt kann eine oder bis zu fünf Projektphasen beinhalten und eine Projektphase gehört immer zu genau einem Projekt. Ein Projekt ohne Projektphasen kann nicht existieren, eine Projektphase ohne Projekt ebenfalls nicht.}\\\hline\hline
    %%%%%%
    Projekt & 1 & 1..* & Teilprojekt\\\hline
    \multicolumn{4}{|p{0.942\textwidth}|}{Ein Projekt kann ein oder mehrere Teilprojekte beinhalten und ein Teilprojekt gehört zu genau einem Projekt. Ein Projekt ohne Teilprojekte kann nicht existieren, ein Teilprojekt ohne Projekt ebenfalls nicht.}\\\hline\hline
    %%%%%%
    Projektphase & 1 & 1..* & Attribut\\\hline
    \multicolumn{4}{|p{0.942\textwidth}|}{Eine Projektphase hat ein oder mehrere Attribute und ein Attribut gehört zu genau einer Projektphase. Eine Projektphase ohne Attribute kann nicht existieren, ein Attribut ohne Projektphase ebenfalls nicht.}\\\hline\hline
    %%%%%%
    Prozess & 1 & 1..* & Prozessschritt\\\hline
    \multicolumn{4}{|p{0.942\textwidth}|}{Ein Prozess hat ein oder mehrere Prozessschritte und ein Prozessschritt gehört immer zu genau einem Prozess. Ein Prozess ohne Prozessschritte kann nicht existieren, ein Prozessschritt ohne Prozess ebenfalls nicht.}\\\hline
    %%%%%%
\end{xltabular}
\captionof{table}[Klassen mit Kompositionen]{Beschreibung der ermittelten Klassen mit Kompositionen}

\subsubsection{Beschreibung der Generalisierungen}
Generalisierungen sind Vererbungsbeziehungen, bei denen es eine Basisklasse gibt, die ihre Attribute an eine oder mehrere spezialisierte, bzw. abgeleitete Klassen vererbt.\\

\begin{xltabular}{\textwidth}{|p{0.472\textwidth}|p{0.472\textwidth}|}
    \hline
    \multicolumn{2}{|c|}{\textbf{Klassen mit Generalisierungen}}\\\hline\hline
    Boolsches Attribut, Textattribut, Zahlenattribut & Attribut \\\hline
    \multicolumn{2}{|p{0.944\textwidth}|}{Die Klasse Attribut ist abstrakt und vererbt ihre Eigenschaften an die abgeleiteten Klassen Boolsches Attribut, Textattribut und Zahlenattribut.}\\\hline\hline
    Administrator, Projektleiter, Teilprojektleiter, Projektmitarbeiter, Kunde & Benutzer \\\hline
    \multicolumn{2}{|p{0.944\textwidth}|}{Die Klasse Benutzer ist abstrakt und vererbt ihre Eigenschaften an die abgeleiteten Klassen Administrator, Projektleiter, Teilprojektleiter, Projektmitarbeiter und Kunde.}\\\hline
\end{xltabular}
\captionof{table}[Klassen mit Generalisierungen]{Beschreibung der ermittelten Klassen mit Generalisierungen}

\subsection{Erweitertes Klassendiagramm}

\subsection{Paketdiagramm}











\begin{comment}
%überarbeiten, an neues Buch anpassen
%Anforderungen --> Anforderungsspezifikation --> Fachliche Lösung
In dem nun folgendem Kapitel wird die Anforderungsanalyse behandelt. Orientiert wird sich dazu an dem Vorgehensmodell von Helmut Balzert, das ausführlich in dem Modul BIS-134 Anforderungsanalyse des Studiengangs Wirtschaftsinformatik der Hochschule Hannover behandelt wurde.\\Die Anforderungsanalyse ist einer der ersten Schritte im Softwareentwicklungsprozess und hat zum Ziel die Anforderungen zu ermitteln, die das System, in diesem Fall der Business Transformation Tracker, leisten soll, sowie diese zu definieren. Dadurch soll eine größtmögliche Abdeckung der gestellten Anforderungen erreicht werden und Unstimmigkeiten mit dem Kunden, bzw. dem Auftraggeber, in Bezug auf Funktion und Umfang, vermieden werden. Im Wasserfallmodell nach Balzert ist die Anforderungsanalyse in der Definitionsphase verortert und arbeitet somit mit den Ergebnisobjekten der vorangegangenen Planungsphase. \footcite[Vgl.][S. 100 ff.]{balzert} Die Ergebnisse der Anforderungsanalyse werden dem anschließenden Kapitel, der Konzeption und somit der Entwurfsphase, als Basis dienen.\\
Im Rahmen dieser    Darstellung in UML


\subsection{Ermittlung der Anforderungen}
Im nachfolgendem Kapitel werden die Anforderungen an die Software, die sich aus der Problemstellung und Gesprächen mit dem Auftraggeber ergeben haben, genauer spezifiziert. Im Anschluss folgen dann die zusätzlichen Anforderungen, die sich aus der Umfrage ergeben haben.

\subsubsection{Nichtfunktionale Anforderungen}
%Anforderungen müssen systematisch gewonnen werden von Beteiligten und Betroffenen, sonstige quellen

\subsection{Spezifizierung der Anforderungen}
%Ermittelte Anforderungen müssen spezifiziert werden, unter Berücksichtigung von festgelegten Methoden, Richtlinien, etc.

\subsection{Analyse der Anforderungen}
%Spezifizierte Anforderungen müssen anhand von Richtlinien und Checklisten analysiert werden

\subsection{Modellierung der Anforderungen}
%analysierte und validierte Anforderungen bilden Ausgangspunkt für Modellierung der fachlichen Lösung

\subsection{Verifikation der Anforderungen}

\subsection{Wahl der Entwicklungsplattform}
warum java, nicht web, nicht abap, nicht etc..
bewertung der it sicherheit anhand bestimmter kriterien (datenschutz, zugriffssicherheit, bewahrung von geschäftsgeheimnissne), java weil protierung auf allen plattformen (windows, unix, macos) verfügbar

\subsection{Pflichtenheft}
Durch den Auftraggeber wurden folgende Anforderungen gestellt:

\subsection{Use-Cases}
Akteure des IT-Systems definieren
Mitarbeiter: Projektmitarbeiter, Projektleiter, Teilprojektleiter
Usecase 1:
Der Projektleiter möchte ein neues Projekt anlegen und die Mitarbeiter zuordnen

Usecase 2:
Der Teilprojektleiter öffnet ein vorhandenes Projekt und fügt erfasst die Prozesse und Subprozesse

Usecase 3:
Ein Projektmitarbeiter möchte den aktuellen Fortschritt in einem Subprozess erfassen.

\subsection{Umgebung}
\subsection{Schnittstellen}

%Methodik
\begin{comment}
    %Methodik
    --> Wasserfallmodell nach Helmut Balzert(1995), S.100 ff.

    Anwendungsfälle
    Geschäftsprozessdiagramm, Aktivitätsdiagramm (Folie 94)
    Anwendungsfalldiagramm, -schablone
    Klassendiagramme --> Beziehungen --> Detailliertes Klassendiagramme
    Attribute Spezifizieren (exemplarisch), Operationen
    Sequenzdiagramm

    Pflichtenheft (genaue spezifizierung) 
    Verfeinerung des Lastenheftes
    Verbale Beschreibung dessen, was das System leisten soll (Auftraggebersicht)
    Dient i. a. als vertragliche Beschreibung des Lieferumfangs
    Einstiegsdokument für alle, die das System später pflegen und warten sollen
    Grundlage für die Erstellung des Produkt-Modells

    Ziel
    •Präzise Festlegung, WAS das System leisten soll (aus Sicht des Auftraggebers)
    Anforderungsanalyse
    •Ermittlung und Beschreibung der Anforderungen des Auftraggebers an ein IT-System
    •Bestimmung dessen, WAS das System leisten soll
    •Erstellen eines logischen Modells

\end{comment}

%GUI-Konzept, Prototyp
\section{Vorstellung Prototyp}
Nachfolgend folgt die Vorstellung des GUI-Prototypen des Business Transformation Trackers. Da dieser als Webanwendung konzeptiert ist, wurde auch der GUI-Protoyp an eine Webanwendung angelehnt. Der Prototyp erfüllt den Zweck die analysierten Anforderungen in einer grafischen Benutzeroberfläche abzubilden, um dem Auftraggeber und den Anwendern erste, leicht verständliche Einblicke in die Entwicklung zu geben. Dieser Prototyp dient einer späteren Umsetzung und Implementierung des BBT als Vorlage für die Benutzeroberfläche.

\subsubsection{Verwendete Werkzeuge}
Für die Entwicklung des Prototypen wurde auf das Programm Adobe XD zurückgegriffen. Dies entstammt dem Softwarehersteller Adobe und ist Teil der Softwaresuite \glqq{}Creative Cloud\grqq{}. Das Programm ist nur in einem Abonnement erhältich und kostet etwa 12 Euro im Monat. Mit Adobe XD ist eine Grafiksoftware mit der sich komplexe, grafische Benutzeroberflächen für unterschiedliche Bildschirmgrößen gestalten und zu einem animierten Prototypen überführen lassen.\footcite[Vgl.][]{adobe}


\subsubsection{Aufbau der Benutzeroberfläche}

%Fazit
\section{Fazit und Ausblick}
Ziel der hiervorliegenden Abschlussarbeit des Studiengangs Wirtschaftsinformatik, war eine Neukonzeption, Datenmodellierung und Prototyperstellung eines Prozess-Tracking-Tools zur Steuerung und Umsetzungsverfolgt einer S/4HANA-Transformation im Vorgehensmodell der adesso orange AG. Dazu wurden zu Beginn die Grundlage von SAP und einer S/4HANA-Transformation ausgiebig vorgestellt. Die S/4HANA-Transformation ist ein komplexer Vorgang, den die meisten Unternehmen, die zum jetzigen Zeitpunkt noch auf SAP-ERP (R/3) setzen, bestreiten müssen. Dazu gibt es den Greenfield- und Brownfield-Ansatz, die zwei extreme Transformationsansätze wiederspiegeln, die jedoch wahrscheinlich nur wenige Unternehmen so bestreiten möchten. Deshalb wurden von unterschiedlichen SAP-Beratungsunternehmen, weitere Transformationsansätze und Vorgehensmodell entwickelt, die einen Kompromiss zu den beiden Ansätzen bieten, und versuchen, dabei das Beste \glqq{}aus beiden Welten\grqq{} zu erhalten. Nach Vorstellung des Unternehmens selbst, wurde das Vorgehensmodell der adesso orange AG ausgiebig vorgestellt. Dieses basiert auf verschiedenen Projektphasen, die jeweils unterschiedliche Bausteine und Methodiken beinhalten, mit denen versucht wird, das Unternehmen des Kunden zu analysieren, und basierend darauf den bestmöglichen Transformationsweg zu entwickeln. 


Rückblickend  lässt sich sagen, dass 

\newpage
\section{Anhang}
\section{Quellenverzeichnis}
\printbibliography
\section{Index}
\section{Erklärung zur ordnungsgemäßen Erstellung}







\end{normalsize}


\end{document}