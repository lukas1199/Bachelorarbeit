\documentclass[12pt, titlepage]{article}
\usepackage[ngerman]{babel}
\usepackage[utf8]{inputenc}
\usepackage{color}
\usepackage[a4paper, lmargin={3cm}, rmargin={2.5cm}, tmargin={3cm}, bmargin={2cm}]{geometry}
\usepackage{amssymb}
\usepackage{amsthm}
\usepackage{graphicx}
\usepackage{helvet}
\usepackage{microtype}
\usepackage{setspace}

\renewcommand{\familydefault}{\sfdefault}
\setlength\parindent{0pt}





\begin{document}

% Leere Titelseite
\pagenumbering{gobble}
\newpage\null\thispagestyle{empty}\newpage

\begin{titlepage}
    \normalsize{Hochschule Hannover, Fakultät IV: Wirtschaft und Informatik

    Bachelorarbeit im Studiengang Wirtschaftsinformatik, Wintersemester 2021/2022} 
    
    \sloppy 
    \textbf{\Large{\\Konzeption, Datenmodellierung und prototypischer Aufbau eines Prozess-Tracking-Tools zur Steuerung und Umsetzungsverfolgung einer S/4HANA Transformation im Vorgehensmodell eines IT-Beratungsunternehmens}}
    \vspace{10cm}
    \normalsize{\\Abgabedatum: 08. Februar 2022 \vspace{1cm}\\Lukas Hampel\\Matrikelnummer: 1481025\\Scharnhorststr. 8\\31785 Hameln\vspace{1cm}\\Erstprüfer: Herr Prof. Dr. Raymond Fleck\\Zweitprüfer: Herr Michael Bloß, adesso orange AG}
\end{titlepage}


\section*{Sperrvermerk}
Lorem
\newpage


\pagenumbering{Roman}
\setcounter{page}{3}
\section*{Vorbemerkung}

\newpage

\tableofcontents

\newpage

\section*{Glossar / Abkürzungsverzeichnis}

\newpage

\listoffigures{}
\listoftables{}

\newpage

\section*{Kurzfassung}

\newpage

\pagenumbering{arabic}
\setcounter{page}{1}
\begin{normalsize}
\linespread{1.5}

% Einleitung
\doublespacing
\section{Einleitung}
\subsection{Motivation}
Die SAP SE (fortan, in Abgrenzung zum Produkt, als "die"\ SAP bezeichnet) ist der größte Anbieter für Unternehmenssoftware in Europa und hat mit dem Produkt SAP-ERP eine der am weitesten verbreiteten Enterprise-Ressource-Planning (ERP)-Software geschaffen. \\Mit der neusten Generation SAP S/4HANA sollen in den nächsten Jahren die bereits etablierten Versionen SAP R/2 und SAP R/3 sukzessive abgelöst werden, bevor die Unterstützung, in Form von Weiterentwicklungen und Aktualisierungen, durch die SAP im Jahr 2030 vollständig eingestellt wird. Die Generation S/4HANA bringt viele neue Funktionen, unter anderem erstmalig Cloud-Funktionalitäten, mit sich, weshalb die Umstellung für die meisten Unternehmen eine große Hürde darstellt, die in der Regel nicht alleine bewältigt werden kann. Allerdings birgt die Aktualisierung auf die neuste Generation auch viele Chancen, um den Aufbau der Systeme und der darin abgebildeten Geschäftsprozesse komplett zu überdenken. Das erleichtert sich bspw. von historisch gewachsenen Strukturen zu trennen um sich dem Industriestandard zu nähern, bzw. ihn zu etablieren. Das verringert im Anschluss an die Umstellung Wartungskosten ermöglicht eine Optimierung der Geschäftsprozesse.\\ Um eine solche Transformation durchzuführen, ist jedoch viel Expertise im Projektmanagement und der Projektorganisation notwendig, vorallem aber auch viel Wissens in den Disziplinen der Fachbereiche.
Die SAP setzt in dem Vertrieb, dem Service, dem Betrieb und der Entwicklung ihrer Produkte auf ein breit aufgestelltes Partnerprogramm, in dem Drittunternehmen aufgenommen werden können, um sich für eine Kooperation zu qualifizieren. Aus diesem Grund haben sich viele IT-Beratungsunternehmen auf das Thema SAP-ERP spezialisiert und bieten nun auch die Umstellung auf die neuste Version für ihre Kunden an.

\subsection{Zielsetzung}
In der hier vorliegenden Bachelorarbeit aus dem Studiengang der Wirtschaftsinformatik soll es um die Konzeption, Datenmodellierung und den protypischen Aufbau eines Prozess-Tracking-Tools gehen, das im Vorgehensmodell eines IT-Beratungsunterneh- mens zur SAP S/4HANA-Transformation zum Einsatz kommen soll.\\ 

\subsection{Vorgehensweise}
Dazu wird zunächst auf die einschlägigen Begrifflichkeiten eingegangen um sich dann dem Themenkomplex der S/4HANA-Transformation zu nähern und ihre Eigentschaften und Besonderheiten zu erklären. Im Anschluss wird zuerst das Unternehmen, in dessen Kontext sich diese Arbeit abspielt, vorgestellt, um dann genauer auf das Geschäftsmodell und das Vorgehensmodell zur S/4HANA Transformation einzugehen. \\
Danach wird der aktuelle Ist-Zustand des Tools, bzw. die Form, die momentan verwendet wird, vorgestellt und genauer darauf eingegangen, warum diese Form durch eine Neuentwicklung ersetzt werden sollte. Schließlich wird die Konzeption des Programms stattfinden. Dazu werden im ersten Schritt die Anforderungen analysiert, indem Interviews mit unterschiedlichen Key-Usern und Stakeholdern geführt werden, um daraus verschiedene Anwendungsfälle und -beispiele heraus zu filtern. In der Anforderungsanalyse werden sich ebenfalls Geschäftsprozessmodelldiagramme, erste Klassendiagramme und Sequenzdiagramme wiederfinden, um die Anforderungen an die Entwicklung zu visualisieren. Im nächsten Schritt werden die erarbeiteten Anforderungen 


\section{Methodik und Vorgehen}

%Grundlagen
\section{Grundlagen}
\subsection{SAP-ERP}
\subsection{Transformationsbegriff}
\subsection{Begrifflichkeiten}


\newpage
\section{Umfeld}
\subsection{Vorstellung des Unternehmens}
\subsection{Einordnung AAT / Notwendigkeit}
\subsection{Notwendigkeit}
\subsection{Aufbau}
\subsection{Phasen}
\subsection{Einordnung des BTT}

\newpage
\section{Vorgehensweise}
\subsection{Zielsetzung}
\subsection{Methodik}
\subsection{Was soll erreicht werden}

\newpage
\section{Erhebung des Ist-Zustand}
\subsection{Was bietet das Tool bereits heute}
\subsection{Welche Verbesserungspotenziale gibt es}
\subsection{Warum verbessern?}
\subsection{Geplante Erweiterungen des Funktionsumfangs}

\newpage
\section{Anforderungsanalyse}
\subsection{Interviews mit Stakeholdern}
\subsection{Use-Cases}
\subsection{Umgebung}
\subsection{Schnittstellen}

\newpage
\section{Datenmodellierung}
\subsection{Aufbau}
\subsection{Beschreibung }
\subsection{...}

\newpage
\section{Konzeption}
\subsection{Datenmodell}
\subsection{Klassen}
\subsection{Beziehungen}
\subsection{...}

\newpage
\section{Prototyp}
\subsection{Aufbau}
\subsection{Beschreibung Funktionalität}
\subsection{Fehlende Feautures}

\newpage
\section{Diskussion}
\subsection{...}

\newpage
\section{Fazit}
\subsection{Messung der Zielerreichung}

\newpage
\section{Schlussteil}


\newpage
\section{Anhang}
\section{Quellenverzeichnis}
\section{Index}
\section{Erklärung zur ordnungsgemäßen Erstellung}







\end{normalsize}


\end{document}