\documentclass[12pt, titlepage]{article}
\usepackage[ngerman]{babel}
\usepackage[utf8]{inputenc}
\usepackage{color}
\usepackage[a4paper, lmargin={3cm}, rmargin={2.5cm}, tmargin={3cm}, bmargin={2cm}]{geometry}
\usepackage{amssymb}
\usepackage{amsthm}
\usepackage{graphicx}
\usepackage{helvet}
\usepackage{microtype}
\renewcommand{\familydefault}{\sfdefault}
\setlength\parindent{0pt}





\begin{document}
% Leere Titelseite
\pagenumbering{gobble}
\newpage\null\thispagestyle{empty}\newpage

\begin{titlepage}
    \normalsize{Hochschule Hannover, Fakultät IV: Wirtschaft und Informatik \\
    Bachelorarbeit im Studiengang Wirtschaftsinformatik, Wintersemester 2021/2022} 
    \vspace{1.5cm}
    \sloppy 
    \textbf{\Large{\\Konzeption, Datenmodellierung und prototypischer Aufbau eines Prozess-Tracking-Tools zur Steuerung und Umsetzungsverfolgung einer S/4HANA Transformation im Vorgehensmodell eines IT-Beratungsunternehmens}}
    \vspace{10cm}
    \normalsize{\\Abgabedatum: 08. Februar 2022 \vspace{1cm}\\Lukas Hampel\\Matrikelnummer: 1481025\\Scharnhorststr. 8\\31785 Hameln\vspace{1cm}\\Erstprüfer: Herr Prof. Dr. Raymond Fleck\\Zweitprüfer: Herr Michael Bloß, adesso orange AG}
\end{titlepage}


\section*{Sperrvermerk}
Lorem
\newpage


\pagenumbering{Roman}
\setcounter{page}{3}
\section*{Vorbemerkung}

\newpage

\tableofcontents

\newpage

\section*{Abkürzungsverzeichnis}

\newpage

\section*{Abbildungs-/Tabellenverzeichnis}

\newpage

\section*{Kurzfassung}

\newpage

\pagenumbering{arabic}
\setcounter{page}{1}
\begin{normalsize}
\linespread{1.5}
\section{Einleitung}

lorem ipsum dolor

\section{Einleitung}
\subsection{Vorstellung des Themas}
\subsection{Einordnung des Themas / Umfeld}

\newpage
\section{Umfeld}
\subsection{Vorstellung des Unternehmens}
\subsection{Einordnung AAT / Notwendigkeit}
\subsection{Notwendigkeit}
\subsection{Aufbau}
\subsection{Phasen}
\subsection{Einordnung des BTT}

\newpage
\section{Vorgehensweise}
\subsection{Zielsetzung}
\subsection{Methodik}
\subsection{Was soll erreicht werden}

\newpage
\section{Erhebung des Ist-Zustand}
\subsection{Was bietet das Tool bereits heute}
\subsection{Welche Verbesserungspotenziale gibt es}
\subsection{Warum verbessern?}
\subsection{Geplante Erweiterungen des Funktionsumfangs}

\newpage
\section{Anforderungsanalyse}
\subsection{Interviews mit Stakeholdern}
\subsection{Use-Cases}
\subsection{Umgebung}
\subsection{Schnittstellen}

\newpage
\section{Konzeption}
\subsection{Datenmodell}
\subsection{Klassen}
\subsection{Beziehungen}
\subsection{...}

\newpage
\section{Prototyp}
\subsection{Aufbau}
\subsection{Beschreibung Funktionalität}
\subsection{Fehlende Feautures}

\newpage
\section{Diskussion}
\subsection{...}

\newpage
\section{Fazit}
\subsection{Messung der Zielerreichung}

\newpage
\section{Schlussteil}


\newpage
\section{Anhang}
\section{Quellenverzeichnis}
\section{Index}
\section{Erklärung zur ordnungsgemäßen Erstellung}







\end{normalsize}


\end{document}