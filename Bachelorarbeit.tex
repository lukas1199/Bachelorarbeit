\documentclass[12pt, titlepage]{article}
\usepackage[ngerman]{babel}
\usepackage[utf8]{inputenc}
\usepackage{color}
\usepackage[a4paper, lmargin={3cm}, rmargin={2.5cm}, tmargin={3cm}, bmargin={2cm}]{geometry}
\usepackage{amssymb}
\usepackage{amsthm}
\usepackage{graphicx}
\usepackage{helvet}
\usepackage{microtype}
\usepackage{setspace}
\usepackage{csquotes}
\usepackage{xpatch}
\usepackage[backend=biber, %% Hilfsprogramm "biber" (statt "biblatex" oder "bibtex")
style=authoryear-ibid, %% Zitierstil (siehe Dokumentation)
natbib=true, %% Bereitstellen von natbib-kompatiblen Zitierkommandos
hyperref=false, %% hyperref-Paket verwenden, um Links zu erstellen
ibidpage=true
]{biblatex}
\addbibresource{literatur.bib}

\renewcommand{\familydefault}{\sfdefault}
\setlength\parindent{0pt}


\begin{document}
% Leere Titelseite
\pagenumbering{gobble}
\newpage\null\thispagestyle{empty}\newpage

\begin{titlepage}
    \normalsize{Hochschule Hannover, Fakultät IV: Wirtschaft und Informatik

    Bachelorarbeit im Studiengang Wirtschaftsinformatik, Wintersemester 2021/2022} 
    
    \sloppy 
    \textbf{\Large{\\Konzeption, Datenmodellierung und prototypischer Aufbau eines Prozess-Tracking-Tools zur Steuerung und Umsetzungsverfolgung einer S/4HANA Transformation im Vorgehensmodell eines IT-Beratungsunternehmens}}
    \vspace{10cm}
    \normalsize{\\Abgabedatum: 08. Februar 2022 \vspace{1cm}\\Lukas Hampel\\Matrikelnummer: 1481025\\Scharnhorststr. 8\\31785 Hameln\vspace{1cm}\\Erstprüfer: Herr Prof. Dr. Raymond Fleck\\Zweitprüfer: Herr Michael Bloß, adesso orange AG}
\end{titlepage}


\section*{Sperrvermerk}
Lorem
\newpage


\pagenumbering{Roman}
\setcounter{page}{3}
\section*{Vorbemerkung}

\newpage

\tableofcontents

\newpage

\section*{Glossar / Abkürzungsverzeichnis}

\newpage

\listoffigures{}
\listoftables{}

\newpage

\section*{Kurzfassung}

\newpage

\pagenumbering{arabic}
\setcounter{page}{1}
\begin{normalsize}
\linespread{1.5}

%Einleitung
\doublespacing
\section{Einleitung}
\subsection{Motivation}
Die SAP SE (fortan, in Abgrenzung zum Produkt, als \glqq{}die\grqq{} SAP bezeichnet) ist der größte Anbieter für Unternehmenssoftware in Europa und hat mit dem Produkt SAP-ERP eine der am weitesten verbreiteten Enterprise-Ressource-Planning (ERP)-Software geschaffen. \\Mit der neusten Generation SAP S/4HANA sollen in den nächsten Jahren die bereits etablierten Versionen SAP R/2 und SAP R/3 sukzessive abgelöst werden, bevor die Unterstützung, in Form von Weiterentwicklungen und Aktualisierungen, durch die SAP im Jahr 2030 vollständig eingestellt wird. Die Generation S/4HANA bringt viele neue Funktionen, unter anderem viele Cloud-Funktionalitäten mit sich, weshalb die Umstellung für die meisten Unternehmen eine große Hürde darstellt, die in der Regel nicht mit den intern vorhandenen Ressourcen bewältigt werden kann. Allerdings bringt die Aktualisierung auf die neuste Generation auch viele Chancen mit sich, um den Aufbau der Systeme und der darin abgebildeten Geschäftsprozesse komplett neu zu denken, 
%NICHT SCHÖN!!!!
da vieles bei der Umstellung sowieso angefasst werden muss. Das erleichtert bspw. die Trennung von historisch gewachsenen Strukturen und die Annäherung bzw. Etablierung des Industriestandards und dessen Best-Practises. Dadurch werden im Anschluss die Wartungskosten für die Systeme verringert und eine Optimierung und Effizienzsteigerung der Geschäftsprozesse erreicht.\\ Um eine solche Transformation durchzuführen, ist jedoch viel Wissen und Erfahrung im Projektmanagement und der Projektorganisation notwendig, vorallem aber auch viel Expertise in den Disziplinen der einzelnen Fachbereiche.
Die SAP setzt in den Bereichen Vertrieb, Service, Betrieb und  Entwicklung ihrer Produkte auf ein breit aufgestelltes Partnerprogramm, in dem Drittunternehmen aufgenommen werden können, um sich für eine Kooperation zu qualifizieren. Dadurch haben sich viele IT-Beratungsunternehmen auf das Themengebiet SAP spezialisiert und bieten nun auch eine SAP S/4HANA-Transformation für ihre Kunden an.

\subsection{Zielsetzung}
In der hier vorliegenden Bachelorarbeit aus dem Studiengang der Wirtschaftsinformatik soll es um die Konzeption, Datenmodellierung und den protypischen Aufbau eines Prozess-Tracking-Tools gehen, das im Vorgehensmodell eines IT-Beratungsunterneh- mens zur SAP S/4HANA-Transformation zum Einsatz kommen soll.\\ 
Das Tool soll auf dem gesamten Transformationspfad in einem S/4HANA-Projekt produktiv zum Einsatz kommen und frühzeitig einen 
Überblick über alle betroffene Transformationsobjekte wiedergeben. 
%nicht schön
Dadurch soll erreicht werden, dass zu jedem Zeitpunkt der aktuelle Fortschritt der Transformation im jeweiligen Prozess wiedergegeben werden kann und kein Artefakt außer acht gelassen wird, wodurch Probleme im späteren Projektverlauf vermieden werden sollen, indem stets alle Aspekte betrachtet werden können.



%Methodik
\section{Methodik und Vorgehen}
\subsection{Methodik}

\subsection{Vorgehen}
%Anfang nochmal ändern 
Dazu wird zunächst auf die einschlägigen Begrifflichkeiten eingegangen um sich dann dem Themenkomplex der S/4HANA-Transformation zu nähern und ihre Eigentschaften und Besonderheiten zu erklären. Im Anschluss wird zuerst das Unternehmen, in dessen Kontext sich diese Arbeit abspielt, vorgestellt, um dann genauer auf das Geschäftsmodell und das Vorgehensmodell zur S/4HANA Transformation einzugehen. \\
Danach wird der aktuelle Ist-Zustand des Tools, bzw. die Form, die momentan verwendet wird, vorgestellt und genauer darauf eingegangen, warum diese Form durch eine Neuentwicklung ersetzt werden sollte. Schließlich wird die Konzeption des Programms stattfinden. Dazu werden im ersten Schritt die Anforderungen analysiert, indem Interviews mit unterschiedlichen Key-Usern und Stakeholdern geführt werden, um daraus verschiedene Anwendungsfälle und -beispiele heraus zu filtern. In der Anforderungsanalyse werden sich ebenfalls Geschäftsprozessmodelldiagramme, erste Klassendiagramme und Sequenzdiagramme wiederfinden, um die Anforderungen an die Entwicklung zu visualisieren. Im nächsten Schritt werden die erarbeiteten Anforderungen ausgeprägt


%Grundlagen
\section{Grundlagen}
\subsection{ERP-Systeme}
ERP ist ein Akronym für den englischen Begriff \glqq{}Enterprise Ressource Planning\grqq{}, also das Planen von Unternehmensressourcen, u.a. in den Bereichen Beschaffung, Produktion, Vertrieb, Personalwirtschaft und Finanzwesen. \footcite[Vgl.][523]{wibuch} Ein ERP-System beschreibt somit eine Software, die Prozesse aus diesen Bereichen in einem Anwendungspaket integriert und die dabei anfallenden Daten in einer zentralen Datenbank abspeichert. Dadurch werden Redundanzen in der Datenhaltung vermieden und bereichsübergreifende Unternehmensprozesse ermöglicht \footcite[Vgl.][523]{wibuch}. ERP-Systeme nutzen in der Regel eine Client-Server-Architektur und sind komponentenorientiert, das heißt, sie können, je nach Anforderungen ihres Wertschöpfungsprozesses, ihre benötigten Komponenten frei wählen. Dadurch ist eine schrittweise Einführung der ERP-Software, über einen längeren Zeitraum, möglich. \footcite[Vgl.][524 f.]{wibuch}

\subsection{SAP}
\subsubsection{Die SAP SE}
Die SAP SE wurde im Jahr 1972 von fünf ehemaligen IBM-Mitarbeitern unter dem Namen \glqq{}\underline{S}ystem\underline{a}nalyse und \underline{P}rogrammentwicklung GbR\grqq{}\footcite[Vgl.][]{think-ing}  mit dem Ziel gegründet, eine Standardanwendungssoftware für die Echtzeitverarbeitung zu entwickeln.  Im Jahr 1973 wurde durch die SAP mit dem \glqq{}System RF\grqq{} das erste Produkt für die Finanzbuchhaltung vorgestellt, was den Grundstein für die erste SAP-Generation \glqq{}SAP R/1\grqq{} legen sollte. Durch die ständigen Weiterentwicklungen wurde das System stets erweitert und fand bei immer mehr Kunden anklang. 1976 wurde die Gesellschaft bürgerlichen Rechts aufeglöst und in eine GmbH überführt. Im selben Jahr wurde bereits mit nur 25 Mitarbeitern ein Umsatz von 3,81 Mio. DM erzielt. \footcite[Vgl.][]{sap-fruehejahre}\\Im Jahr 1979 folgt schließlich die zweite Produktgeneration \glqq{}SAP R/2\grqq{}

\subsubsection{SAP-ERP}

\subsubsection{SAP HANA}

\subsubsection{S/4HANA}

\subsection{Transformation}



\newpage
\section{Umfeld}
\subsection{Vorstellung des Unternehmens}
\subsection{Geschäftsmodell}
\subsubsection{Beispiel Kunde}
\subsection{Einordnung AAT / Notwendigkeit}
\subsection{Notwendigkeit}
\subsection{Aufbau}
\subsection{Phasen}
\subsection{Einordnung des BTT}


%Ist-Zustand
\section{Erhebung des Ist-Zustand}

\subsection{Was bietet das Tool bereits heute}
Zum jetzigen Zeitpunkt exisitiert der Business-Transformation-Tracker bereits in Form eines Excel-Spreadsheets. Dieses wird bereits in einigen Transformationsprojekten des betrachteten Unternehmens verwendet und unterstützt dadurch schon heute die Mitarbeiter in den Projekten. \\Das Tool dient dazu in einem S/4HANA Transformationsprojekt 
\subsection{Welche Verbesserungspotenziale gibt es}

\subsection{Warum verbessern?}

\subsection{Geplante Erweiterungen des Funktionsumfangs}

\subsection{Interviews mit Stakeholdern}

%Anforderungsanalyse
\section{Anforderungsanalyse}
In dem nun folgendem Kapitel wird die Anforderungsanalyse behandelt. Orientiert wird sich dazu an dem Vorgehensmodell von Helmut Balzert, das ausführlich in dem Modul BIS-134 Anforderungsanalyse des Studiengangs Wirtschaftsinformatik der Hochschule Hannover behandelt wurde.\\Die Anforderungsanalyse ist einer der ersten Schritte im Softwareentwicklungsprozess und hat zum Ziel die Anforderungen zu ermitteln, die das System, in diesem Fall der Business Transformation Tracker, leisten soll, sowie diese zu definieren. Dadurch soll eine größtmögliche Abdeckung der gestellten Anforderungen erreicht werden und Unstimmigkeiten mit dem Kunden, bzw. dem Auftraggeber, in Bezug auf Funktion und Umfang, vermieden werden. Im Wasserfallmodell nach Balzert ist die Anforderungsanalyse in der Definitionsphase verortert und arbeitet somit mit den Ergebnisobjekten der vorangegangenen Planungsphase. \footcite[Vgl.][S. 100 ff.]{balzert} Die Ergebnisse der Anforderungsanalyse werden dem anschließenden Kapitel, der Konzeption und somit der Entwurfsphase, als Basis dienen.\\
Im Rahmen dieser    Darstellung in UML


--> Wasserfallmodell nach Helmut Balzert(1995), S.100 ff.

Anwendungsfälle
Geschäftsprozessdiagramm, Aktivitätsdiagramm (Folie 94)
Anwendungsfalldiagramm, -schablone
Klassendiagramme --> Beziehungen --> Detailliertes Klassendiagramme
Attribute Spezifizieren (exemplarisch), Operationen
Sequenzdiagramm

Pflichtenheft (genaue spezifizierung) 
Verfeinerung des Lastenheftes
Verbale Beschreibung dessen, was das System leisten soll (Auftraggebersicht)
Dient i. a. als vertragliche Beschreibung des Lieferumfangs
Einstiegsdokument für alle, die das System später pflegen und warten sollen
Grundlage für die Erstellung des Produkt-Modells

Ziel
•Präzise Festlegung, WAS das System leisten soll (aus Sicht des Auftraggebers)
Anforderungsanalyse
•Ermittlung und Beschreibung der Anforderungen des Auftraggebers an ein IT-System
•Bestimmung dessen, WAS das System leisten soll
•Erstellen eines logischen Modells

\subsection{Pflichtenheft}
\subsection{Use-Cases}
Akteure des IT-Systems definieren
Mitarbeiter: Projektmitarbeiter, Projektleiter, Teilprojektleiter
Usecase 1:
Der Projektleiter möchte ein neues Projekt anlegen und die Mitarbeiter zuordnen

Usecase 2:
Der Teilprojektleiter öffnet ein vorhandenes Projekt und fügt erfasst die Prozesse und Subprozesse

Usecase 3:
Ein Projektmitarbeiter möchte den aktuellen Fortschritt in einem Subprozess erfassen.

\subsection{Umgebung}
\subsection{Schnittstellen}

\newpage
\section{Datenmodellierung}
\subsection{Aufbau}
\subsection{Beschreibung }
\subsection{...}

\newpage
\section{Konzeption}

--> Ziel
•Präzise Festlegung, WIE das Fachkonzept softwaretechnisch umgesetzt werden soll
Entwurf der Software-Architektur
•Technische Grobstruktur des Systems
Entwurf der Anwendungs-Architektur
•Zerlegung des Gesamtsystems in fachlich zusammengehörige Teile

\subsection{Datenmodell}
\subsection{Klassen}
\subsection{Beziehungen}
\subsection{...}

\newpage
\section{Prototyp}
Implementierungsphase
Ziel
•Realisierung des Systems in Form von Programmen
Programmierung
Testen
•einzelne Komponenten
•Gesamtsystem

\subsection{Aufbau}
\subsection{Beschreibung Funktionalität}
\subsection{Fehlende Feautures}

\newpage
\section{Diskussion}
\subsection{...}

\section{Reflexion}

\newpage
\section{Fazit}
\subsection{Messung der Zielerreichung}



\newpage
\section{Schlussteil}


\newpage
\section{Anhang}
\section{Quellenverzeichnis}
\printbibliography
\section{Index}
\section{Erklärung zur ordnungsgemäßen Erstellung}







\end{normalsize}


\end{document}