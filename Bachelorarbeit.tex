\documentclass[12pt, titlepage]{article}
\usepackage[ngerman]{babel}
\usepackage[utf8]{inputenc}
\usepackage{color}
\usepackage[a4paper, lmargin={3cm}, rmargin={2.5cm}, tmargin={3cm}, bmargin={2cm}]{geometry}
\usepackage{amssymb}
\usepackage{amsthm}
\usepackage{graphicx}
\usepackage{helvet}
\usepackage{microtype}
\usepackage{setspace}

\renewcommand{\familydefault}{\sfdefault}
\setlength\parindent{0pt}





\begin{document}

% Leere Titelseite
\pagenumbering{gobble}
\newpage\null\thispagestyle{empty}\newpage

\begin{titlepage}
    \normalsize{Hochschule Hannover, Fakultät IV: Wirtschaft und Informatik \\
    Bachelorarbeit im Studiengang Wirtschaftsinformatik, Wintersemester 2021/2022} 
    \vspace{1.5cm}
    \sloppy 
    \textbf{\Large{\\Konzeption, Datenmodellierung und prototypischer Aufbau eines Prozess-Tracking-Tools zur Steuerung und Umsetzungsverfolgung einer S/4HANA Transformation im Vorgehensmodell eines IT-Beratungsunternehmens}}
    \vspace{10cm}
    \normalsize{\\Abgabedatum: 08. Februar 2022 \vspace{1cm}\\Lukas Hampel\\Matrikelnummer: 1481025\\Scharnhorststr. 8\\31785 Hameln\vspace{1cm}\\Erstprüfer: Herr Prof. Dr. Raymond Fleck\\Zweitprüfer: Herr Michael Bloß, adesso orange AG}
\end{titlepage}


\section*{Sperrvermerk}
Lorem
\newpage


\pagenumbering{Roman}
\setcounter{page}{3}
\section*{Vorbemerkung}

\newpage

\tableofcontents

\newpage

\section*{Abkürzungsverzeichnis}

\newpage

\section*{Abbildungs-/Tabellenverzeichnis}

\newpage

\section*{Kurzfassung}

\newpage

\pagenumbering{arabic}
\setcounter{page}{1}
\begin{normalsize}
\linespread{1.5}

%Einleitung
\doublespacing
\section{Einleitung}
\subsection{Motivation}
Die SAP SE (fortan, in Abgrenzung zum Produkt, als \glqq{}die\grqq{} SAP bezeichnet) ist der größte Anbieter für Unternehmenssoftware in Europa \footcite[Vgl.][]{sap-about} und hat mit dem Produkt SAP-ERP eine der am weitesten verbreiteten Enterprise-Ressource-Planning (ERP)-Software geschaffen \footcite[Vgl.][]{sap-about}. \\Mit der neusten Generation SAP S/4HANA sollen in den nächsten Jahren die bereits etablierten Versionen SAP R/2 und SAP R/3 sukzessive abgelöst werden, bevor die Unterstützung, in Form von Weiterentwicklungen und Aktualisierungen, durch die SAP bis zum Jahr 2030 vollständig eingestellt wird\footcite[Vgl.][]{sap-support}. Die neuste Generation, SAP S/4HANA, bringt viele neue Funktionen mit sich, unter anderem eine neue Datenbanktechnologie und Cloud-Lösungen, weshalb die Umstellung für die meisten Unternehmen eine große Hürde darstellt, die in der Regel nicht mit den intern vorhandenen Ressourcen bewältigt werden kann. Allerdings bringt die Aktualisierung auf die neuste Generation auch viele Chancen mit sich, um den Aufbau der Systeme und den darin abgebildeten Geschäftsprozesse komplett neu zu denken, da durch die Änderungen und neuen Funktionen üblicherweise viele Prozesse bei der Umstellung überarbeitet werden müssen. Das erleichtert beispielsweise die Abtrennung von historisch gewachsenen Strukturen und die Annäherung bzw. die Etablierung des Industriestandards und dessen Best Practices. Dadurch erreicht man im Anschluss eine Verringerung der Wartungskosten für die Systeme und eine Optimierung und Effizienzsteigerung der Geschäftsprozesse.\\ 
Die SAP setzt in den Bereichen Vertrieb, Service, Betrieb und  Entwicklung ihrer Produkte auf ein breit aufgestelltes Partnerprogramm, in dem Drittunternehmen aufgenommen werden können, um sich für eine Kooperation zu qualifizieren \footcite[Vgl.][]{sap-partner}. Dadurch haben sich viele IT-Beratungsunternehmen auf das Themengebiet SAP spezialisiert und bieten nun auch eine SAP S/4HANA-Transformation für ihre Kunden an. Um eine S/4HANA-Transformation durchzuführen, ist viel Wissen und Erfahrung im Projektmanagement und der Projektorganisation notwendig, vor allem aber auch viel Expertise in den Disziplinen der einzelnen Fachbereiche. Dabei kommen viele unterschiedliche Methodiken und Tools zum Einsatz, die den Projektmitarbeitern die Arbeit erleichtern und ihnen stets eine Übersicht über die bereits geleistete Arbeit und den noch zu erledigenden Aufgaben geben sollen. 

\subsection{Zielsetzung und Vorgehen}
In der hier vorliegenden Bachelorarbeit aus dem Studiengang der Wirtschaftsinformatik soll es um die Konzeption, Datenmodellierung und den prototypischen Aufbau eines Prozess-Tracking-Tools gehen, das im Vorgehensmodell eines IT-Beratungsunternehmens zur SAP S/4HANA-Transformation zum Einsatz kommen soll.\\ Das Tool soll auf dem gesamten Transformationspfad eines S/4HANA-Projekts produktiv zum Einsatz kommen und frühzeitig einen Überblick über alle betroffenen Geschäftsprozesse geben und den aktuellen Transformationsfortschritt dieser, bis zur Fertigstellung des Projekts, erfassen und wiedergeben können. Dadurch soll erreicht werden, dass zu jedem Zeitpunkt in einem Projekt, der aktuelle Transformationsfortschritt eines Geschäftsprozesses begutachtet werden kann. Durch die ganzheitliche Erfassung Prozessen soll erreicht werden, dass keine Details oder Bestandteile außer Acht gelassen werden, wodurch Probleme im Projektverlauf vermieden und die später notwendige Fehlerbehebung auf ein Minimum reduziert werden soll. Das erklärte Ziel ist dabei die Qualitätsverbesserung und die Effizienzsteigerung der S/4HANA-Transformation.\\
In dieser Arbeit soll es in erster Linie um die Konzeptionierung dieses Tools gehen. Dazu wird der Requirements Engineering-Prozess bis zu der Erreichung eines GUI-Prototyps umgesetzt und eine Datenmodellierung in Form von UML-Diagrammen angefertigt. Bis mit der Konzeptionierung jedoch begonnen wird, werden zuerst die theoretischen Grundlagen zu SAP und zur S/4HANA-Transformation beschrieben und im Anschluss der unternehmerische Kontext beschrieben, in dessen die Systemkonzeption stattfindet. Dazu wird das auftraggebende Unternehmen vorgestellt, sein Geschäftsmodell beschrieben und das selbst entwickelte Vorgehensmodell zur S/4HANA-Transformation erklärt. Im Anschluss beginnt der Konzeptionierungsprozess mit der Erhebung des Ist-Zustandes, in der die Problemstellung dargestellt und die aktuell im Einsatz befindliche Lösung beschrieben wird. Danach erfolgt, nach Analyse der verschiedenen Stakeholder, die Spezifikation der Anforderungen, die im Anschluss mithilfe einer objektorientierten Analyse und eines Datenmodells analysiert werden. Ergebnis dieses Requirements Engineering-Prozesses ist eine fachliche Lösung mit einem OOA-Modell (Objektorientiertes-Analyse-Modell) und einem Oberflächen-Prototyp, der die Datenelemente des OOA-Modells auf der Benutzeroberfläche abbildet.\\Nach Beendigung des Analyseprozess folgt zum Schluss das abschließende Fazit mit einem Ausblick. 



%Methodik
\section{Methodik und Vorgehen}
\subsection{Methodik}

\subsection{Vorgehen}
%Anfang nochmal ändern 
Dazu wird zunächst auf die einschlägigen Begrifflichkeiten eingegangen um sich dann dem Themenkomplex der S/4HANA-Transformation zu nähern und ihre Eigentschaften und Besonderheiten zu erklären. Im Anschluss wird zuerst das Unternehmen, in dessen Kontext sich diese Arbeit abspielt, vorgestellt, um dann genauer auf das Geschäftsmodell und das Vorgehensmodell zur S/4HANA Transformation einzugehen. \\
Danach wird der aktuelle Ist-Zustand des Tools, bzw. die Form, die momentan verwendet wird, vorgestellt und genauer darauf eingegangen, warum diese Form durch eine Neuentwicklung ersetzt werden sollte. Schließlich wird die Konzeption des Programms stattfinden. Dazu werden im ersten Schritt die Anforderungen analysiert, indem Interviews mit unterschiedlichen Key-Usern und Stakeholdern geführt werden, um daraus verschiedene Anwendungsfälle und -beispiele heraus zu filtern. In der Anforderungsanalyse werden sich ebenfalls Geschäftsprozessmodelldiagramme, erste Klassendiagramme und Sequenzdiagramme wiederfinden, um die Anforderungen an die Entwicklung zu visualisieren. Im nächsten Schritt werden die erarbeiteten Anforderungen ausgeprägt


%Grundlagen
\section{Grundlagen}
\subsection{ERP-Systeme}
ERP ist ein Akronym für den englischen Begriff \glqq{}Enterprise Ressource Planning\grqq{}, also das Planen von Unternehmensressourcen, u.a. in den Bereichen Beschaffung, Produktion, Vertrieb, Personalwirtschaft und Finanzwesen. \footcite[Vgl.][523]{wibuch} Ein ERP-System beschreibt somit eine Software, die Prozesse aus diesen Bereichen in einem Anwendungspaket integriert und die dabei anfallenden Daten in einer zentralen Datenbank abspeichert. Dadurch werden Redundanzen in der Datenhaltung vermieden und bereichsübergreifende Unternehmensprozesse ermöglicht \footcite[Vgl.][523]{wibuch}. ERP-Systeme nutzen in der Regel eine Client-Server-Architektur und sind komponentenorientiert, das heißt, sie können, je nach Anforderungen ihres Wertschöpfungsprozesses, ihre benötigten Komponenten frei wählen. Dadurch ist eine schrittweise Einführung der ERP-Software, über einen längeren Zeitraum, möglich. \footcite[Vgl.][524 f.]{wibuch}

\subsection{SAP}
\subsubsection{Die SAP SE}
Die SAP SE wurde im Jahr 1972 von fünf ehemaligen IBM-Mitarbeitern unter dem Namen \glqq{}\underline{S}ystem\underline{a}nalyse und \underline{P}rogrammentwicklung GbR\grqq{}\footcite[Vgl.][]{think-ing}  mit dem Ziel gegründet, eine Standardanwendungssoftware für die Echtzeitverarbeitung zu entwickeln.  Im Jahr 1973 wurde durch die SAP mit dem \glqq{}System RF\grqq{} das erste Produkt für die Finanzbuchhaltung vorgestellt, was den Grundstein für die erste SAP-Generation \glqq{}SAP R/1\grqq{} legen sollte. Durch die ständigen Weiterentwicklungen wurde das System stets erweitert und fand bei immer mehr Kunden anklang. 1976 wurde die Gesellschaft bürgerlichen Rechts aufeglöst und in eine GmbH überführt. Im selben Jahr wurde bereits mit nur 25 Mitarbeitern ein Umsatz von 3,81 Mio. DM erzielt. \footcite[Vgl.][]{sap-fruehejahre}\\Im Jahr 1979 folgt schließlich die zweite Produktgeneration \glqq{}SAP R/2\grqq{}

\subsubsection{SAP-ERP}

\subsubsection{SAP HANA}

\subsubsection{S/4HANA}

\subsection{Transformation}
\subsubsection{Definition}
Unter einer Transformation versteht man im allgemeinen einen grundlegenden Wandel, der durch bestimmte Faktoren, wie z.B. einer sprunghafte wirtschaftlichen, oder technologischen Entwicklung hervorgerufen wird. Die Transformation hält dabei idR. über einen längeren Zeitraum an und ist erst beendet, sobald sich die neu geschaffenen Strukturen etabliert und gefestigt haben.\footcite[Vgl.][]{difu}\\ Im betriebswirtschaftlichen Kontext versteht man unter einer Transformation (oder auch Business Transformation) die gezielte Umgestaltung eines Unternehmens und seiner Geschäftsprozesse, um auf veränderte Bedingungen am Markt einzugehen und sich ihnen anzupassen. Dabei ist das Ziel durch effizientere und vereinfachte Geschäftsprozesse einen Mehrwert in Form von niedrigeren Kosten bei gleichbleibender, oder bestenfalls verbesserter Qualität zu erreichen und dabei zusätzlich die Kundenzufriedenheit zu steigern.\footcite[Vgl.][]{leanix}

\subsubsection{Die vier R der Transformation}
In den 1990er-Jahren wurde durch Gouillart und Kelly das Modell der \glqq{}Vier R der Transformation\grqq{} \footcite[Vgl.][]{4r-modell} entwickelt, was eine mögliche Form der Business Transformation darstellen soll. Aus diesem Modell hat die Beratungsgesellschaft Gemini Consulting (später in der Capgemini SE aufgegangen)\footcite[Vgl.][]{gemini-died} ein Produkt entwickelt, indem die vier R für vier verschiedene Transformationsdimensionen stehen:\\
\begin{figure}[h]
    \centering
    \includegraphics[scale=0.5]{Bilder/businesstransformationManagementportal.png}
    \caption[]{Die vier R der Transformation}
\end{figure}
\begin{itemize}
    \item[] \emph{Reframing (dt. Einstellungsveränderung)} soll in einem Unternehmen dazu beitragen die Sichtweise auf sich selbst zu überdenken um sich dadurch von alten Denkmustern zu befreien. Um diese Einstellungsveränderung anzustoßen ist es wichtig, dass die Mitarbeiter motiviert werden und davon überzeugt sind durch die eingesetze Energie einen Mehrwert zu generieren. Im nächsten Schritt muss anschließend eine Vision definiert werden, die sich erheblich von der präsenten Realität absetzt um im Anschluss daraus Ziele und Messgrößen zu entwickeln. 
    \item[] \emph{Restructuring (dt. Restrukturierung)} 
    \item[] \emph{Revitalising (dt. Wiederbelebung)}
    \item[] \emph{Renewing (dt. Erneuerung)}
\end{itemize}

\subsubsection{Digitale Transformation}
-- Digitale Transformation
\subsubsection{Besonderheiten der S/4HANA Transformation}



\newpage
\section{Umfeld}
\subsection{Vorstellung des Unternehmens}
\subsection{Einordnung AAT / Notwendigkeit}
\subsection{Notwendigkeit}
\subsection{Aufbau}
\subsection{Phasen}
\subsection{Einordnung des BTT}

\newpage
\section{Vorgehensweise}
\subsection{Zielsetzung}
\subsection{Methodik}
\subsection{Was soll erreicht werden}

\newpage
\section{Erhebung des Ist-Zustand}
\subsection{Was bietet das Tool bereits heute}
\subsection{Welche Verbesserungspotenziale gibt es}
\subsection{Warum verbessern?}
\subsection{Geplante Erweiterungen des Funktionsumfangs}

\newpage
\section{Anforderungsanalyse}
\subsection{Interviews mit Stakeholdern}
\subsection{Use-Cases}
\subsection{Umgebung}
\subsection{Schnittstellen}

\newpage
\section{Datenmodellierung}
\subsection{Aufbau}
\subsection{Beschreibung }
\subsection{...}

\newpage
\section{Konzeption}
\subsection{Datenmodell}
\subsection{Klassen}
\subsection{Beziehungen}
\subsection{...}

\newpage
\section{Prototyp}
\subsection{Aufbau}
\subsection{Beschreibung Funktionalität}
\subsection{Fehlende Feautures}

\newpage
\section{Diskussion}
\subsection{...}

\newpage
\section{Fazit}
\subsection{Messung der Zielerreichung}

\newpage
\section{Schlussteil}


\newpage
\section{Anhang}
\section{Quellenverzeichnis}
\section{Index}
\section{Erklärung zur ordnungsgemäßen Erstellung}







\end{normalsize}


\end{document}