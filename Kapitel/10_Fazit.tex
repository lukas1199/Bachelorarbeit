\section{Fazit und Ausblick}
Ziel der hiervorliegenden Abschlussarbeit des Studiengangs Wirtschaftsinformatik, war eine Neukonzeption, Datenmodellierung und Prototyperstellung eines Prozess-Tracking-Tools zur Steuerung und Umsetzungsverfolgt einer S/4HANA-Transformation im Vorgehensmodell der adesso orange AG. Dazu wurden zu Beginn die Grundlage von SAP und einer S/4HANA-Transformation ausgiebig vorgestellt. Die S/4HANA-Transformation ist ein komplexer Vorgang, den die meisten Unternehmen, die zum jetzigen Zeitpunkt noch auf SAP-ERP (R/3) setzen, bestreiten müssen. Dazu gibt es den Greenfield- und Brownfield-Ansatz, die zwei extreme Transformationsansätze wiederspiegeln, die jedoch wahrscheinlich nur wenige Unternehmen so bestreiten möchten. Deshalb wurden von unterschiedlichen SAP-Beratungsunternehmen, weitere Transformationsansätze und Vorgehensmodell entwickelt, die einen Kompromiss zu den beiden Ansätzen bieten, und versuchen, dabei das Beste \glqq{}aus beiden Welten\grqq{} zu erhalten. Nach Vorstellung des Unternehmens selbst, wurde das Vorgehensmodell der adesso orange AG ausgiebig vorgestellt. Dieses basiert auf verschiedenen Projektphasen, die jeweils unterschiedliche Bausteine und Methodiken beinhalten, mit denen versucht wird, das Unternehmen des Kunden zu analysieren, und basierend darauf den bestmöglichen Transformationsweg zu entwickeln. 


Rückblickend  lässt sich sagen, dass 