\section{Fazit und Ausblick}
Ziel der hiervorliegenden Abschlussarbeit des Studiengangs Wirtschaftsinformatik, war eine Neukonzeption, Datenmodellierung und Prototyperstellung eines Prozess-Tracking-Tools zur Steuerung und Umsetzungsverfolgung einer S/4HANA-Transformation im Vorgehensmodell der adesso orange AG. Dazu wurden zu Beginn die theoretischen Grundlagen zu SAP und einer S/4HANA-Transformation ausgiebig vorgestellt. Die S/4HANA-Transformation ist ein komplexer Vorgang, den die meisten Unternehmen, die zum jetzigen Zeitpunkt noch auf SAP-ERP (R/3) setzen, bestreiten müssen. Dazu gibt es den Greenfield- und Brownfield-Ansatz, die zwei extreme Transformationsansätze wiederspiegeln, die wahrscheinlich nur wenige Unternehmen ohne Abänderungen in Anspruch nehmen möchten. Darum wurden von unterschiedlichen SAP-Beratungsunternehmen, weitere Transformationsansätze und Vorgehensmodelle entwickelt, die einen Kompromiss zu den beiden extremen Ansätzen bieten und dabei versuchen, das \glqq{}Beste aus beiden Welten\grqq{} zu erhalten.\\Nach der Vorstellung des Unternehmens selbst, wurde das Vorgehensmodell der adesso orange AG ausgiebig dargestellt. Dieses basiert auf verschiedenen Projektphasen, die jeweils unterschiedliche Bausteine und Methodiken beinhalten, mit denen das Unternehmen des Kunden zu analysiert wird und basierend darauf der bestmöglichen Transformationsweg entwickelt wird. Ein wichtiger Bestandteil dieses Vorgehensmodell ist der Business Transformation Tracker, der das eingangsbesagte Prozess-Tracking-Tool verkörpert, aber in seiner jetzigen Implementierung Probleme mit sich bringt. Um diese Probleme aufzuzeigen und in einer Neukonzeption zu beheben, wurde zuerst der Ist-Zustand der jetzigen Umsetzung analysiert und bewertet. Parralel dazu fanden Gespräche mit dem Auftraggeber von adesso orange und die Durchführung einer Umfrage, indenen die Mitarbeiter von adesso orange zu ihre Meinung zum Business Transformation Tracker abgeben konnten, statt. Dabei wurde zu dem Schluss gekommen, dass eine Neuentwicklung des BTT ratsam ist, um die Zufriedenheit der Mitarbeiter zu steigern und um von weiteren Problemen, wie der mangelnden Übersichtlichkeit, Abstand zu nehmen. 
\\Im nächsten Schritt der Konzeptionierung erfolgte die Anforderungsermittlung, in der zuerst die einzelnen Stakeholder des Systems analysiert wurden. Danach wurden die Anforderungen aus dem vorhergehenden Kapitel ermittelt und mithilfe der bereits erwähnten Umfrage und persönlichen Gesprächen, Verbesserungsvorschläge und die Wünsche der Mitarbeiter gesammelt. Im Anschluss wurden die ermittelten funktionalen Anforderungen genauer spezifiziert, indem sie systematisch detailliert und durch die Aufsetzung von verschiedenen Anwendungsfälle zusammengetragen wurden. Diese Anwendungsfälle wurden tabellarisch dargestellt und durch Aktivitätsdiagramme dargestellt. Ebenfalls wurde im selben Zug die nichtfunktionalen Anforderungen erfasst, sowie die Rahmenbedingungen und der Kontext. Anschließend begann mit den gesammelten Anforderungen die Modellierung der Daten. Dazu wurden die benötigten Klassen herausgerarbeitet und mit ihren jeweiligen Beziehungen in einem Übersichtsklassendiagramm dargestellt und beschrieben. Als nächstes wurde das Übersichtsklassendiagramm zu einem erweiterten Klassendiagramm ausgebaut, in dem zusätzlich die Attribute und Operationen der Klassen dargestellt wurden. Zuletzt wurden die Klassen in einem Paketdiagramm zu verschiedenen Paketen zusammengefasst.
Als letztes folgte die Vorstellung eines Prototypen, der die ermittelten OOA-Element in einer grafischen Benutzeroberfläche abbildet. Dieser dient der Evaluierung des objektorientierten Analysemodells mit den zukünfitgen Benutzern und dem Auftraggeber, der im nächsten Schritt entscheiden muss, ob die Implementierung begonnen wird, und wenn ja, welche Änderungen noch vorgenommen werden müssen.\\
Ausgehend von dem jetzigen Stand, lässt sich sagen, dass eine Fortführung der Neuentwicklung sinnvoll ist, um mit einer ausgereiften Anwendung, die bisherige Umsetzung des Business Transformation Trackers abzulösen. Dies stellt zwar einen nicht zu vernachlässigbaren Aufwand dar, jedoch wird die Zahl der S/4HANA-Transformationen in den nächsten fünf Jahren vermutlich weiter steigen, da das Support-Ende von SAP-ERP durch die SAP SE immer näher rückt und somit viele noch nicht transformierte Unternehmen immer mehr unter Druck stehen, die S/4HANA-Transformation in den nächsten Jahren zu vollführen.