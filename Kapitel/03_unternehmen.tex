\section{Unternehmerischer Kontext}
\subsection{Die adesso orange AG}
\subsubsection{Vorstellung des Unternehmens}
Die adesso orange AG ist ein IT-Beratungsunternehmen, das sich vorallem auf die SAP-Beratung spezialisert hat. Sie ist ein Tochterunternehmen der Dortmunder adesso SE, das im Jahr 2021 aus einer Fusion der adesso SE mit der in Hameln ansässigen Quanto AG hervorging. Die adesso SE ist eines der größten IT-Dienstleistungs- und Beratungsunternehmen Deutschlands und hat ca. 5600 Mitarbeiter an 44 Standorten in ganz Deutschland und Europa. \footcite[Vgl.][]{adesso-main} \\Die adesso SE wurde im Jahr 1997 als \glqq{}adesso Beratungsgesellschaft für Software-Prozeß-Management mbH\grqq{} gegründet und hatte Ende der 1990er Jahre erste größere Projekte im Versicherungssektor. Im Jahr 2000 wurde die Gesellschaft mit beschränkter Haftung schließlich zu einer Aktiengesellschaft umgewandelt, zu diesem Zeitpunkt hatte sie bereits 100 Mitarbeiter. Ende der 2000er-Jahre hatte die adesso bereits über 500 Mitarbeiter und erschloss zunehmend die internationalen Märkte in ganz Europa. Im Jahr 2012 erwirtschaftete die adesso AG mit ca. 1000 Mitarbeitern über 100 Mio. Euro Umsatz und vergrößerte sich in den nachfolgenden Jahren durch das Gründen von Tochtergesellschaften stetig. Im Jahr 2019 wurde die adesso AG in eine europäische Aktiengesellschaft \glqq{}Societas Europaea\grqq{} (SE) umgewandelt und war mit über 4000 Mitarbeitern, laut \glqq{}Lünendonk-Liste 2020\grqq{} das größte mittelständische IT-Beratungsunternehmen in Deutschland.\footcite[Vgl.][]{adesso-historie} Im Geschäftsjahr 2020 erreichte die adesso SE eine Umsatzsteigerung von 16 Prozent, im Vergleich zum Vorjahr, auf 523,375 Mio. Euro, wovon ca. 413 Mio. Euro in Deutschland und 110 Mio. Euro im Ausland erwirtschaftet wurden. Von den 523 Mio. Euro waren 60,406 Mio. Euro als Gewinn (EBITDA) zu verzeichnen, was eine Steigerung von 26 Prozent gegenüber dem Vorjahr entspricht. Nach dem Abzug der Abgaben verblieb für das Jahr 2020 ein Konzernergebnis von ca. 21 Mio. Euro.\footcite[Vgl.][S. 4]{adesso2020-report} Besonders sind dabei die Auswirkung der von Anfang 2020 bis dato anhaltenen Covid-19-Pandemie hevorzuheben, die zeitweise zu unterjährigen Wachstumseinbrüchen von bis zu 9,8 Prozent führten. Dies resultierte aus der neuaufgetretenen, allgemeinen Unsicherheit, die unter anderem dazu führte, das adesso-Kunden Projekte stoppten oder verschoben. Das führte dazu, dass Gesellschaften der adesso SE, wie auch viele andere Unternehmen, im Zeitraum von April bis Juli 2020, teilweise Kurzarbeit anmelden mussten und Maßnahmen zur Liquiditätssicherung, wie zum Beispiel der Vereinbarung von Steuerstundungen, veranlassten. Da es sich bei vielen Kunden von adesso um Versicherungen oder Banken handelt, die dem öffentlichen Sektor enstammen, waren die Auswirkungen der Pandemie nicht allzu prikär, was zu einer Erholung im zweiten Halbjahr führte.\footcite[Vgl.][S. 30f]{adesso2020-report} Im zweiten Halbjahr 2020 wurde auch der Mehrheitserwerb an der Quanto AG in Höhe von 71,4 Prozent durchgeführt, der zu der Fusion und später, im Jahr 2021, zu der Neugründung der adesso orange AG führte.\footcite[Vgl.][S. 15]{adesso2020-report} 
\\Mit der adesso orange AG hat die adesso SE den auf SAP spezialiserten Teil ihres Unternehmens, zusammen mit der ehemaligen Quanto AG, in einem seperaten Unternehmen gebündelt, das sich vorallem auf die SAP-Beratung von Banken, Energieversorgern und Versicherungen spezialisiert hat.\footcite[Vgl.][]{ao-main} Derzeit beschäftigt die adesso orange AG ca. 300 Mitarbeiter an den Standorten der adesso SE. Der Hauptsitz befindet sich weiterhin, wie bereits bei der Quanto AG, in Hameln.\footcite[Vgl.][]{ao-karriere}
Die bis 2021 bestehende Quanto AG ging im Jahr 2016 aus dem Zusammenschluss der Firmen \glqq{}Aequitas\grqq{} und \glqq{}Quanto\grqq{} hervor, um gemeinsam größere Kunden zu gewinnen.\footcite[Vgl.][]{ww-quanto} Die Quanto AG hielt bereits Standorte in Hamburg, Kiel, Flensburg, Stuttgart, Heidelberg, Berlin sowie im ungarischen Budapest und Györ und hatte bereits im Jahr 2016 ca. 140 Mitarbeiter. Neben den Bereichen der SAP konzentrierte sich die Quanto AG auch auf das \glqq{}Internet der Dinge\grqq{} und Blockchain-Technologien.\footcite[Vgl.][]{ww-quanto}

\subsubsection{Geschäftsmodell}
Bei der adesso orange AG handelt es sich um ein Beratungsunternehmen, das sich auf die SAP-Beratung im öffentlichen Sektor spezialisiert hat. 

\subsection{Vorgehensmodell S/4HANA Transformation}
\subsubsection{Aufbau}

\subsubsection{Phasen}

\subsubsection{Tools}

\subsubsection{Methodiken}

\subsubsection{Einordnung des BTT}