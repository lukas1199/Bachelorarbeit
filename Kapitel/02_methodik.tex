\section{Methodik und Vorgehen}
\subsection{Methodik}
\subsubsection{Vorgehensmodell nach Balzert}
Umfrage der MA
\subsection{Vorgehen}
%Anfang nochmal ändern 
Dazu wird zunächst auf die einschlägigen Begrifflichkeiten eingegangen um sich dann dem Themenkomplex der S/4HANA-Transformation zu nähern und ihre Eigentschaften und Besonderheiten zu erklären. Im Anschluss wird zuerst das Unternehmen, in dessen Kontext sich diese Arbeit abspielt, vorgestellt, um dann genauer auf das Geschäftsmodell und das Vorgehensmodell zur S/4HANA Transformation einzugehen. \\
Danach wird der aktuelle Ist-Zustand des Tools, bzw. die Form, die momentan verwendet wird, vorgestellt und genauer darauf eingegangen, warum diese Form durch eine Neuentwicklung ersetzt werden sollte. Schließlich wird die Konzeption des Programms stattfinden. Dazu werden im ersten Schritt die Anforderungen analysiert, indem Interviews mit unterschiedlichen Key-Usern und Stakeholdern geführt werden, um daraus verschiedene Anwendungsfälle und -beispiele heraus zu filtern. In der Anforderungsanalyse werden sich ebenfalls Geschäftsprozessmodelldiagramme, erste Klassendiagramme und Sequenzdiagramme wiederfinden, um die Anforderungen an die Entwicklung zu visualisieren. Im nächsten Schritt werden die erarbeiteten Anforderungen ausgeprägt
