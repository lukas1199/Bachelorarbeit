\section{Einleitung}
\subsection{Motivation}
Die SAP SE (fortan, in Abgrenzung zum Produkt, als \glqq{}die\grqq{} SAP bezeichnet) ist der größte Anbieter für Unternehmenssoftware in Europa \footcite{sap-about} und hat mit dem Produkt SAP-ERP eine der am weitesten verbreiteten Enterprise-Ressource-Planning (ERP)-Software geschaffen \footcite{}. \\Mit der neusten Generation SAP S/4HANA sollen in den nächsten Jahren die bereits etablierten Versionen SAP R/2 und SAP R/3 sukzessive abgelöst werden, bevor die Unterstützung, in Form von Weiterentwicklungen und Aktualisierungen, durch die SAP bis zum Jahr 2030 vollständig eingestellt wird\footcite{sap-support}. Die neuste Generation, SAP S/4HANA, bringt viele neue Funktionen mit sich, unter anderem eine neue Datenbanktechnologie und Cloud-Lösungen, weshalb die Umstellung für die meisten Unternehmen eine große Hürde darstellt, die in der Regel nicht mit den intern vorhandenen Ressourcen bewältigt werden kann. Allerdings bringt die Aktualisierung auf die neuste Generation auch viele Chancen mit sich, um den Aufbau der Systeme und der darin abgebildeten Geschäftsprozesse komplett neu zu denken, da durch die Änderungen und neuen Funktionen üblicherweise viele Prozesse bei der Umstellung überarbeitet werden müssen. Das erleichtert beispielsweise die Trennung von historisch gewachsenen Strukturen und die Annäherung bzw. die Etablierung des Industriestandards und dessen Best Practices. Dadurch erreicht man im Anschluss eine Verringerung der Wartungskosten für die Systeme und eine Optimierung und Effizienzsteigerung der Geschäftsprozesse.\footcite{}\\ 
Die SAP setzt in den Bereichen Vertrieb, Service, Betrieb und  Entwicklung ihrer Produkte auf ein breit aufgestelltes Partnerprogramm, in dem Drittunternehmen aufgenommen werden können, um sich für eine Kooperation zu qualifizieren \footcite{sap-partner}. Dadurch haben sich viele IT-Beratungsunternehmen auf das Themengebiet SAP spezialisiert und bieten nun auch eine SAP S/4HANA-Transformation für ihre Kunden an. Um eine S/4HANA-Transformation durchzuführen, ist viel Wissen und Erfahrung im Projektmanagement und der Projektorganisation notwendig, vor allem aber auch viel Expertise in den Disziplinen der einzelnen Fachbereiche. Dabei kommen viele unterschiedliche Methodiken und Tools zum Einsatz, die den Projektmitarbeitern die Arbeit erleichtern soll und ihnen stets eine Übersicht über die bereits geleistete Arbeit und den noch zu erledigenden Aufgaben geben soll. 

\subsection{Zielsetzung und Vorgehen}
In der hier vorliegenden Bachelorarbeit aus dem Studiengang der Wirtschaftsinformatik soll es um die Konzeption, Datenmodellierung und den prototypischen Aufbau eines Prozess-Tracking-Tools gehen, das im Vorgehensmodell eines IT-Beratungsunterneh- mens zur SAP S/4HANA-Transformation zum Einsatz kommen soll.\\ Das Tool soll auf dem gesamten Transformationspfad eines S/4HANA-Projekts produktiv zum Einsatz kommen und frühzeitig einen Überblick über alle betroffenen Geschäftsprozesse geben und den aktuellen Transformationsfortschritt dieser, bis zur Fertigstellung des Projekts, erfassen und wiedergeben können. Dadurch soll erreicht werden, dass zu jedem Zeitpunkt in einem Projekt der aktuelle Transformationsfortschritt eines Geschäftsprozesses begutachtet werden kann. Durch die ganzheitliche Erfassung Prozessen soll erreicht werden, dass keine Details oder Bestandteile außer Acht gelassen werden, wodurch Probleme im Projektverlauf vermieden und die später notwendige Fehlerbehebung auf ein Minimum reduziert werden soll. Das erklärte Ziel ist dabei die Qualitätsverbesserung und die Effizienzsteigerung der S/4HANA-Transformation.\\
In dieser Arbeit soll es in erster Linie um die Konzeptionierung dieses Tools gehen. Dazu wird der Requirements Engineering-Prozess bis zu der Erreichung eines GUI-Prototyps umgesetzt und eine Datenmodellierung in Form von UML-Diagrammen angefertigt. Bis mit der Konzeptionierung jedoch begonnen wird, werden zuerst die theoretischen Grundlagen zu SAP und zur S/4HANA-Transformation beschrieben und im Anschluss der unternehmerische Kontext beschrieben, in dessen die Systemkonzeption stattfindet. Dazu wird das auftraggebende Unternehmen vorgestellt, sein Geschäftsmodell beschrieben und das selbst entwickelte Vorgehensmodell zur S/4HANA-Transformation erklärt. Im Anschluss beginnt der Konzeptionierungsprozess mit der Erhebung des Ist-Zustandes, in der die Problemstellung dargestellt wird und die aktuell im Einsatz befindliche Lösung beschrieben wird. Danach erfolgt, nach Analyse der verschiedenen Stakeholder, die Spezifikation der Anforderungen, die im Anschluss mithilfe einer objektorientierten Analyse und eines Datenmodells analysiert werden. Ergebnis dieses Requirements Engineering-Prozesses ist eine fachliche Lösung mit einem OOA-Modell (Objektorientiertes-Analyse-Modell) und einem ersten Oberflächen-Prototyps, der die Datenelemente des OOA-Modells auf der Benutzeroberfläche abbildet.\\Nach Beendigung der Anforderungsanalyse folgt zum Schluss das abschließende Fazit mit einer Diskussion, in der das Vorgehen nocheinmal bewertet wird. 

