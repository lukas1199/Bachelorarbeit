\section{Einleitung}
\subsection{Motivation}
Die SAP SE (fortan, in Abgrenzung zum Produkt, als "die"\ SAP bezeichnet) ist der größte Anbieter für Unternehmenssoftware in Europa und hat mit dem Produkt SAP-ERP eine der am weitesten verbreiteten Enterprise-Ressource-Planning (ERP)-Software geschaffen. \\Mit der neusten Generation SAP S/4HANA sollen in den nächsten Jahren die bereits etablierten Versionen SAP R/2 und SAP R/3 sukzessive abgelöst werden, bevor die Unterstützung, in Form von Weiterentwicklungen und Aktualisierungen, durch die SAP im Jahr 2030 vollständig eingestellt wird. Die Generation S/4HANA bringt viele neue Funktionen, unter anderem viele Cloud-Funktionalitäten mit sich, weshalb die Umstellung für die meisten Unternehmen eine große Hürde darstellt, die in der Regel nicht mit den intern vorhandenen Ressourcen bewältigt werden kann. Allerdings bringt die Aktualisierung auf die neuste Generation auch viele Chancen mit sich, um den Aufbau der Systeme und der darin abgebildeten Geschäftsprozesse komplett neu zu denken, 
%NICHT SCHÖN!!!!
da vieles bei der Umstellung sowieso angefasst werden muss. Das erleichtert bspw. die Trennung von historisch gewachsenen Strukturen und die Annäherung bzw. Etablierung des Industriestandards und dessen Best-Practises. Dadurch werden im Anschluss die Wartungskosten für die Systeme verringert und eine Optimierung und Effizienzsteigerung der Geschäftsprozesse erreicht.\\ Um eine solche Transformation durchzuführen, ist jedoch viel Wissen und Erfahrung im Projektmanagement und der Projektorganisation notwendig, vorallem aber auch viel Expertise in den Disziplinen der einzelnen Fachbereiche.
Die SAP setzt in den Bereichen Vertrieb, Service, Betrieb und  Entwicklung ihrer Produkte auf ein breit aufgestelltes Partnerprogramm, in dem Drittunternehmen aufgenommen werden können, um sich für eine Kooperation zu qualifizieren. Dadurch haben sich viele IT-Beratungsunternehmen auf das Themengebiet SAP spezialisiert und bieten nun auch eine SAP S/4HANA-Transformation für ihre Kunden an.

\subsection{Zielsetzung}
In der hier vorliegenden Bachelorarbeit aus dem Studiengang der Wirtschaftsinformatik soll es um die Konzeption, Datenmodellierung und den protypischen Aufbau eines Prozess-Tracking-Tools gehen, das im Vorgehensmodell eines IT-Beratungsunterneh- mens zur SAP S/4HANA-Transformation zum Einsatz kommen soll.\\ 
Das Tool soll auf dem gesamten Transformationspfad in einem S/4HANA-Projekt produktiv zum Einsatz kommen und frühzeitig einen 
Überblick über alle betroffene Transformationsobjekte wiedergeben. 
%nicht schön
Dadurch soll erreicht werden, dass zu jedem Zeitpunkt der aktuelle Fortschritt der Transformation im jeweiligen Prozess wiedergegeben werden kann und kein Artefakt außer acht gelassen wird, wodurch Probleme im späteren Projektverlauf vermieden werden sollen, indem stets alle Aspekte betrachtet werden können.

