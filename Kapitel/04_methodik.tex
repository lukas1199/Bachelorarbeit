\section{Methodik}
In diesem Kapitel wird das Vorgehen zu einer Neukonzeptionierung des Business Transformation Trackers erklärt.

\subsection{Untersuchung des Ist-Zustandes}
Zu Beginn findet die Vorstellung des Problems statt, die der Business Transformation Tracker lösen soll und in der erklärt wird, welchen Nutzen und Funktionen er dazu bieten soll. Danach erfolgt die Beschreibung des Ist-Zustandes, in der die aktuelle Implementierung analysiert und der Aufbau beschrieben wird. Im Anschluss findet eine Bewertung der aktuellen Implementierung statt, in der ihre Probleme aufgezeigt und analysiert wird, welche Vor- und Nachteile sie hat. Am Ende wird eine Empfehlung gegeben, wie eine sinnvolle Neuentwicklung aussehen sollte.

\subsection{Requirements Engineering}
Nach der Analyse des Ist-Zustandes folgt die Anforderungserhebung an das neue System. Dazu werden zuerst die einzelnen Stakeholder des BTT analysiert und vorgestellt. Es wird eine Risikoermittlung durchgeführt, in der dargestellt wird, welche potentiellen Auswirkungen auf das Vorhaben möglich sind. Danach wird das Vorgehen zur Anforderungserhebung beschrieben. Diese geschieht durch persönliche Gespräche mit dem Auftraggeber und durch eine Onlinebefragung bei adesso orange, in der die Mitarbeiter nach ihrer Meinung zur aktuellen Implementierung und zu Verbesserungsvorschlägen und Wünschen befragt werden. Im Anschluss erfolgt die Auswertung der Umfrage, in der das aktuelle Meinungsbild beschrieben wird und die Forderungen und Wünsche zusammengefasst dargestellt werden. Schließlich folgt die Spezifikation der Anforderungen, die sich an einer von Helmut Balzert, in seinem Buch \glqq{}Lehrbuch der Softwaretechnik: Basiskonzepte und Requirements Engineering\grqq{}, vorgestellten Anforderungsschablone orientiert. Jedoch wird dabei kein seprates Lasten- und Pflichtenheft erstellt, sondern diese in einer \glqq{}Requirements Specification\grqq{} zusammengefasst. Die Spezifikation der funktionalen Anforderungen erfolgt außerdem zusammengefasst in Anwendungsfällen, die die Arbeitsvorgänge im System darstellen. Diese Anwendungsfälle werden ebenfalls in Aktivitätsdiagrammen dargestellt, um ihre Abläufe zu visualisieren. 

\subsection{Entwicklung der fachlichen Lösung}
Nach der Anforderungsspezifikation erfolgt eine Datenmodellierung, indem aus den Anwendungsfällen Klassen ermittelt und diese im Anschluss durch Attribute, Multiplizitäten und Operation weiter spezifiziert werden. Dazu wird ein Übersichtsklassen-diagramm und ein erweitertes Klassendiagramm erstellt, sowie ein Paketdiagramm angefertigt, das die Klassen zu Paketen zusammenfasst.

\subsection{Prototypentwicklung}
Nachdem die Konzeption und die Datenmodellierung abgeschlossen sind, erfolgt die Entwicklung eines Oberflächen-Prototyps, der die grafische Benutzeroberfläche des Business Transformation Tracker darstellen. In diesem Prototyp werden die definierten Klassen und Beziehungen abgebildet sowie die vorgestellten Anwendungsfälle visualisiert. Dies dient der Vorlage einer späteren Umsetzung und Implementierung des hier vorgestellten Konzeptes. 