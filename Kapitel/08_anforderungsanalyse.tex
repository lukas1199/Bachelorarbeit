

\begin{comment}
        
%überarbeiten, an neues Buch anpassen
%Anforderungen --> Anforderungsspezifikation --> Fachliche Lösung
In dem nun folgendem Kapitel wird die Anforderungsanalyse behandelt. Orientiert wird sich dazu an dem Vorgehensmodell von Helmut Balzert, das ausführlich in dem Modul BIS-134 Anforderungsanalyse des Studiengangs Wirtschaftsinformatik der Hochschule Hannover behandelt wurde.\\Die Anforderungsanalyse ist einer der ersten Schritte im Softwareentwicklungsprozess und hat zum Ziel die Anforderungen zu ermitteln, die das System, in diesem Fall der Business Transformation Tracker, leisten soll, sowie diese zu definieren. Dadurch soll eine größtmögliche Abdeckung der gestellten Anforderungen erreicht werden und Unstimmigkeiten mit dem Kunden, bzw. dem Auftraggeber, in Bezug auf Funktion und Umfang, vermieden werden. Im Wasserfallmodell nach Balzert ist die Anforderungsanalyse in der Definitionsphase verortert und arbeitet somit mit den Ergebnisobjekten der vorangegangenen Planungsphase. \footcite[Vgl.][S. 100 ff.]{balzert} Die Ergebnisse der Anforderungsanalyse werden dem anschließenden Kapitel, der Konzeption und somit der Entwurfsphase, als Basis dienen.\\
Im Rahmen dieser    Darstellung in UML


\subsection{Ermittlung der Anforderungen}
Im nachfolgendem Kapitel werden die Anforderungen an die Software, die sich aus der Problemstellung und Gesprächen mit dem Auftraggeber ergeben haben, genauer spezifiziert. Im Anschluss folgen dann die zusätzlichen Anforderungen, die sich aus der Umfrage ergeben haben.

\subsubsection{Nichtfunktionale Anforderungen}
%Anforderungen müssen systematisch gewonnen werden von Beteiligten und Betroffenen, sonstige quellen

\subsection{Spezifizierung der Anforderungen}
%Ermittelte Anforderungen müssen spezifiziert werden, unter Berücksichtigung von festgelegten Methoden, Richtlinien, etc.

\subsection{Analyse der Anforderungen}
%Spezifizierte Anforderungen müssen anhand von Richtlinien und Checklisten analysiert werden

\subsection{Modellierung der Anforderungen}
%analysierte und validierte Anforderungen bilden Ausgangspunkt für Modellierung der fachlichen Lösung

\subsection{Verifikation der Anforderungen}

\subsection{Wahl der Entwicklungsplattform}
warum java, nicht web, nicht abap, nicht etc..
bewertung der it sicherheit anhand bestimmter kriterien (datenschutz, zugriffssicherheit, bewahrung von geschäftsgeheimnissne), java weil protierung auf allen plattformen (windows, unix, macos) verfügbar

\subsection{Pflichtenheft}
Durch den Auftraggeber wurden folgende Anforderungen gestellt:

\subsection{Use-Cases}
Akteure des IT-Systems definieren
Mitarbeiter: Projektmitarbeiter, Projektleiter, Teilprojektleiter
Usecase 1:
Der Projektleiter möchte ein neues Projekt anlegen und die Mitarbeiter zuordnen

Usecase 2:
Der Teilprojektleiter öffnet ein vorhandenes Projekt und fügt erfasst die Prozesse und Subprozesse

Usecase 3:
Ein Projektmitarbeiter möchte den aktuellen Fortschritt in einem Subprozess erfassen.

\subsection{Umgebung}
\subsection{Schnittstellen}

%Methodik
\begin{comment}
    %Methodik
    --> Wasserfallmodell nach Helmut Balzert(1995), S.100 ff.

    Anwendungsfälle
    Geschäftsprozessdiagramm, Aktivitätsdiagramm (Folie 94)
    Anwendungsfalldiagramm, -schablone
    Klassendiagramme --> Beziehungen --> Detailliertes Klassendiagramme
    Attribute Spezifizieren (exemplarisch), Operationen
    Sequenzdiagramm

    Pflichtenheft (genaue spezifizierung) 
    Verfeinerung des Lastenheftes
    Verbale Beschreibung dessen, was das System leisten soll (Auftraggebersicht)
    Dient i. a. als vertragliche Beschreibung des Lieferumfangs
    Einstiegsdokument für alle, die das System später pflegen und warten sollen
    Grundlage für die Erstellung des Produkt-Modells

    Ziel
    •Präzise Festlegung, WAS das System leisten soll (aus Sicht des Auftraggebers)
    Anforderungsanalyse
    •Ermittlung und Beschreibung der Anforderungen des Auftraggebers an ein IT-System
    •Bestimmung dessen, WAS das System leisten soll
    •Erstellen eines logischen Modells

\end{comment}