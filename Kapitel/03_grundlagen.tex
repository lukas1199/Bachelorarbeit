\section{Grundlagen}
\subsection{ERP-Systeme}
ERP ist ein Akronym für den englischen Begriff \glqq{}Enterprise Ressource Planning\grqq{}, also das Planen von Unternehmensressourcen, u.a. in den Bereichen Beschaffung, Produktion, Vertrieb, Personalwirtschaft und Finanzwesen. \footcite[Vgl.][523]{wibuch} Ein ERP-System beschreibt somit eine Software, die Prozesse aus diesen Bereichen in einem Anwendungspaket integriert und die dabei anfallenden Daten in einer zentralen Datenbank abspeichert. Dadurch werden Redundanzen in der Datenhaltung vermieden und bereichsübergreifende Unternehmensprozesse ermöglicht \footcite[Vgl.][523]{wibuch}. ERP-Systeme nutzen in der Regel eine Client-Server-Architektur und sind komponentenorientiert, das heißt, sie können, je nach Anforderungen ihres Wertschöpfungsprozesses, ihre benötigten Komponenten frei wählen. Dadurch ist eine schrittweise Einführung der ERP-Software, über einen längeren Zeitraum, möglich. \footcite[Vgl.][524 f.]{wibuch}

\subsection{SAP}
\subsubsection{Die SAP SE}
Die SAP SE wurde im Jahr 1972 von fünf ehemaligen IBM-Mitarbeitern unter dem Namen \glqq{}\underline{S}ystem\underline{a}nalyse und \underline{P}rogrammentwicklung GbR\grqq{}\footcite[Vgl.][]{think-ing}  mit dem Ziel gegründet, eine Standardanwendungssoftware für die Echtzeitverarbeitung zu entwickeln.  Im Jahr 1973 wurde durch die SAP mit dem \glqq{}System RF\grqq{} das erste Produkt für die Finanzbuchhaltung vorgestellt, was den Grundstein für die erste SAP-Generation \glqq{}SAP R/1\grqq{} legen sollte. Durch die ständigen Weiterentwicklungen wurde das System stets erweitert und fand bei immer mehr Kunden anklang. 1976 wurde die Gesellschaft bürgerlichen Rechts aufeglöst und in eine GmbH überführt. Im selben Jahr wurde bereits mit nur 25 Mitarbeitern ein Umsatz von 3,81 Mio. DM erzielt. \footcite[Vgl.][]{sap-fruehejahre}\\Im Jahr 1979 folgt schließlich die zweite Produktgeneration \glqq{}SAP R/2\grqq{}

\subsubsection{SAP-ERP}

\subsubsection{SAP HANA}

\subsubsection{S/4HANA}

\subsection{Transformation}
\subsubsection{Definition}
Unter einer Transformation versteht man im allgemeinen einen grundlegenden Wandel, der durch bestimmte Faktoren, wie z.B. einer sprunghafte wirtschaftlichen, oder technologischen Entwicklung hervorgerufen wird. Die Transformation hält dabei idR. über einen längeren Zeitraum an und ist erst beendet, sobald sich die neu geschaffenen Strukturen etabliert und gefestigt haben.\footcite[Vgl.][]{difu}\\ Im betriebswirtschaftlichen Kontext versteht man unter einer Transformation (oder auch Business Transformation) die gezielte Umgestaltung eines Unternehmens und seiner Geschäftsprozesse, um auf veränderte Bedingungen am Markt einzugehen und sich ihnen anzupassen. Dabei ist das Ziel durch effizientere und vereinfachte Geschäftsprozesse einen Mehrwert in Form von niedrigeren Kosten bei gleichbleibender, oder bestenfalls verbesserter Qualität zu erreichen und dabei zusätzlich die Kundenzufriedenheit zu steigern.\footcite[Vgl.][]{leanix}

\subsubsection{Die vier R der Transformation}
In den 1990er-Jahren wurde durch Gouillart und Kelly das Modell der \glqq{}Vier R der Transformation\grqq{} \footcite[Vgl.][]{4r-modell} entwickelt, was eine mögliche Form der Business Transformation darstellen soll. Aus diesem Modell hat die Beratungsgesellschaft Gemini Consulting (später in der Capgemini SE aufgegangen)\footcite[Vgl.][]{gemini-died} ein Produkt entwickelt, indem die vier R für vier verschiedene Transformationsdimensionen stehen:\\
\begin{figure}[h]
    \centering
    \includegraphics[scale=0.5]{Bilder/businesstransformationManagementportal.png}
    \caption[]{Die vier R der Transformation}
\end{figure}
\begin{itemize}
    \item[] \emph{Reframing (dt. Einstellungsveränderung)} soll in einem Unternehmen dazu beitragen die Sichtweise auf sich selbst zu überdenken um sich dadurch von alten Denkmustern zu befreien. Um diese Einstellungsveränderung anzustoßen ist es wichtig, dass die Mitarbeiter motiviert werden und davon überzeugt sind durch die eingesetze Energie einen Mehrwert zu generieren. Im nächsten Schritt muss anschließend eine Vision definiert werden, die sich erheblich von der präsenten Realität absetzt um im Anschluss daraus Ziele und Messgrößen zu entwickeln. 
    \item[] \emph{Restructuring (dt. Restrukturierung)} 
    \item[] \emph{Revitalising (dt. Wiederbelebung)}
    \item[] \emph{Renewing (dt. Erneuerung)}
\end{itemize}

\subsubsection{Digitale Transformation}
-- Digitale Transformation
\subsubsection{Besonderheiten der S/4HANA Transformation}
