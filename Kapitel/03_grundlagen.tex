\section{Grundlagen}
\subsection{ERP-Systeme}
ERP ist ein Akronym für den englischen Begriff \glqq{}Enterprise Ressource Planning\grqq{}, also das Planen von Unternehmensressourcen, u.a. in den Bereichen Beschaffung, Produktion, Vertrieb, Personalwirtschaft und Finanzwesen. \footcite[Vgl.][523]{wibuch} Ein ERP-System beschreibt somit eine Software, die Prozesse aus diesen Bereichen in einem Anwendungspaket integriert und die dabei anfallenden Daten in einer zentralen Datenbank abspeichert. Dadurch werden Redundanzen in der Datenhaltung vermieden und bereichsübergreifende Unternehmensprozesse ermöglicht \footcite[Vgl.][523]{wibuch}. ERP-Systeme nutzen in der Regel eine Client-Server-Architektur und sind komponentenorientiert, das heißt, sie können, je nach Anforderungen ihres Wertschöpfungsprozesses, ihre benötigten Komponenten frei wählen. Dadurch ist eine schrittweise Einführung der ERP-Software, über einen längeren Zeitraum, möglich. \footcite[Vgl.][524 f.]{wibuch}

\subsection{SAP}
\subsubsection{Die SAP SE}
Die SAP SE wurde im Jahr 1972 von fünf ehemaligen IBM-Mitarbeitern unter dem Namen \glqq{}\underline{S}ystem\underline{a}nalyse und \underline{P}rogrammentwicklung GbR\grqq{}\footcite[Vgl.][]{think-ing}  mit dem Ziel gegründet, eine Standardanwendungssoftware für die Echtzeitverarbeitung zu entwickeln.  Im Jahr 1973 wurde durch die SAP mit dem \glqq{}System RF\grqq{} das erste Produkt für die Finanzbuchhaltung vorgestellt, was den Grundstein für die erste SAP-Generation \glqq{}SAP R/1\grqq{} legen sollte. Durch die ständigen Weiterentwicklungen wurde das System stets erweitert und fand bei immer mehr Kunden anklang. 1976 wurde die Gesellschaft bürgerlichen Rechts aufeglöst und in eine GmbH überführt. Im selben Jahr wurde bereits mit nur 25 Mitarbeitern ein Umsatz von 3,81 Mio. DM erzielt. \footcite[Vgl.][]{sap-fruehejahre}\\Im Jahr 1979 folgt schließlich die zweite Produktgeneration \glqq{}SAP R/2\grqq{}

\subsubsection{SAP-ERP}

\subsubsection{SAP HANA}

\subsubsection{S/4HANA}

\subsection{Transformation}
