\section{Grundlagen}
\subsection{Begrifflichkeiten}
\subsubsection{ERP}
ERP ist ein Akronym für den englischen Begriff \glqq{}Enterprise Ressource Planning\grqq{}, also das Planen von Unternehmensressourcen, u.a. in den Bereichen Beschaffung, Produktion, Vertrieb, Personalwirtschaft und Finanzwesen \footcite[Vgl.][]{wibuch}. Ein ERP-System beschreibt somit eine Software, die Prozesse aus diesen Bereichen in einem Anwendungspaket integriert und die dabei anfallenden Daten in einer zentralen Datenbank abspeichert. Dadurch werden Redundanzen in der Datenhaltung vermieden und bereichsübergreifende Unternehmensprozesse ermöglicht.

\subsection{SAP-ERP}
\subsection{Transformationsbegriff}
