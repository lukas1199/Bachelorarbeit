\section{Einleitung}
Die SAP SE (fortan, in Abgrenzung zum Produkt, als "die"\ SAP bezeichnet) ist der größte Anbieter für Unternehmenssoftware in Europa und hat mit dem Produkt SAP-ERP eine der am weitesten verbreiteten Enterprise-Ressource-Planning (ERP)-Software geschaffen. \\Mit der neusten Generation SAP S/4HANA sollen in den nächsten Jahren die bereits etablierten Versionen SAP R/2 und SAP R/3 sukzessive abgelöst werden, bevor die Unterstützung, in Form von Weiterentwicklungen und Aktualisierungen, durch die SAP im Jahr 2030 vollständig eingestellt wird. Die Generation S/4HANA bringt viele neue Funktionen, unter anderem erstmalig Cloud-Funktionalitäten, mit sich, weshalb die Umstellung für die meisten Unternehmen eine große Hürde darstellt, die in der Regel nicht alleine bewältigt werden kann. Die SAP setzt in dem Vertrieb, dem Service, dem Betrieb und der Entwicklung ihrer Produkte auf ein breit aufgestelltes Partnerprogramm, in dem Drittunternehmen aufgenommen werden können, um sich für eine Kooperation zu qualifizieren. Aus diesem Grund haben sich viele IT-Beratungsunternehmen auf das Thema SAP-ERP spezialisiert und bieten nun auch die Umstellung auf die neuste Version für ihre Kunden an.
\\\\
In der hier vorliegenden Bachelorarbeit aus dem Studiengang Wirtschaftsinformatik soll es um die Konzeption, Datenmodellierung und den protypischen Aufbau eines Prozess-Tracking-Tools gehen, das im Vorgehensmodell eines IT-Beratungsunterneh- mens zur SAP S/4HANA-Transformation zum Einsatz kommen soll.\\ Dazu wird zunächst auf die einschlägigen Begrifflichkeiten eingegangen um sich dann dem Themenkomplex der S/4HANA-Transformation zu nähern und ihre Eigentschaften und Besonderheiten zu erklären. Im Anschluss wird zuerst das Unternehmen, in dessen Kontext sich diese Arbeit abspielt, vorgestellt, um dann genauer auf das Geschäftsmodell und das Vorgehensmodell zur S/4HANA Transformation einzugehen. \\

