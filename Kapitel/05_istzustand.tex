\section{Erhebung des Ist-Zustand}

\subsection{Problemstellung}
Der Business-Transformation-Tracker ist ein wichtiger Bestandteil der adesso orange AG eigenen Vorgehensweise zur Durchführung von S/4HANA Transformationsprojekten. Er soll dazu dienen alle im SAP-System abgebildeten Geschäftsprozesse inkl. ihrer Subprozesse und Prozessschritte zu erfassen, um sie entlang des Transformationspfades zu begleiten und übersichtlich den Fortschritt der Transformation zu ermitteln. Dadurch möchte man erreichen, dass alle abgebildeten Geschäftsprozesse unter den selben Gesichtspunkten betrachtet und bewertet werden, wodurch zum einen die Transformation der Prozesse hinterher wiedergegeben und zurückverfolgt werden kann und zum anderem für die Projektleitung stets ein aktueller Fortschrittsgrad eingeholt werden kann. Auch dient der BTT als Verzeichnis für alle den Prozess betreffende Informationen, wie z.B. die Beschreibung im Fachkonzept, aber auch als Checkliste für vorzunehmende Maßnahmen, wie z.B. der Aktualisierung des Benutzerhandbuchs.  


\subsection{Ist-Zustand}
Zum jetzigen Zeitpunkt exisitiert der Business-Transformation-Tracker bereits in Form eines Excel-Spreadsheets. Dieses wird bereits in einigen Transformationsprojekten des betrachteten Unternehmens verwendet und unterstützt dadurch schon heute die Mitarbeiter in den Projekten. \\Das Tool dient dazu in einem S/4HANA Transformationsprojekt 

\subsection{Ergebnisse der Befragung}

\subsection{Anforderungen}
-Umfrageergebnisse

Voruntersuchung
•Finden und Definieren des Problems
•Ist-Analyse
•Durchführbarkeits-Untersuchung
-technisch, wirtschaftlich

Aufgaben
•Ist-Situation analysieren
•Hauptanforderungen zusammenstellen
•Lösungsvarianten betrachten
•Empfehlung aussprechen
•Projektkalkulation erstellen
•Projektplan vorschlagen

Ergebnisdokumente
•Lastenheft und Glossar
•Projektkalkulation und Projektplan

Lastenheft (themensammlung)
Erste schriftliche Abstimmung zwischen Auftraggeber und Auftragnehmer über fachliche Basisanforderungen.
Notwendig, da Aufträge typischerweise unvollständig und widersprüchlich sowie unterschiedlich interpretierbar sind.
Wird später zum Pflichtenheft erweitert (daher auch: „grobes Pflichtenheft“)
Im Englischen „Requirements Specification“ (keine Unterscheidung von Lasten-/Pflichtenheft)

\subsection{Welche Verbesserungspotenziale gibt es}

\subsection{Warum verbessern?}

\subsection{Geplante Erweiterungen des Funktionsumfangs}

\subsection{Interviews mit Stakeholdern}