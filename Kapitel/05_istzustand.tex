\section{Erhebung des Ist-Zustand}

\subsection{Was bietet das Tool bereits heute}
Zum jetzigen Zeitpunkt exisitiert der Business-Transformation-Tracker bereits in Form eines Excel-Spreadsheets. Dieses wird bereits in einigen Transformationsprojekten des betrachteten Unternehmens verwendet und unterstützt dadurch schon heute die Mitarbeiter in den Projekten. \\Das Tool dient dazu in einem S/4HANA Transformationsprojekt 

Voruntersuchung
•Finden und Definieren des Problems
•Ist-Analyse
•Durchführbarkeits-Untersuchung
-technisch, wirtschaftlich

Aufgaben
•Ist-Situation analysieren
•Hauptanforderungen zusammenstellen
•Lösungsvarianten betrachten
•Empfehlung aussprechen
•Projektkalkulation erstellen
•Projektplan vorschlagen

Ergebnisdokumente
•Lastenheft und Glossar
•Projektkalkulation und Projektplan

Lastenheft (themensammlung)
Erste schriftliche Abstimmung zwischen Auftraggeber und Auftragnehmer über fachliche Basisanforderungen.
Notwendig, da Aufträge typischerweise unvollständig und widersprüchlich sowie unterschiedlich interpretierbar sind.
Wird später zum Pflichtenheft erweitert (daher auch: „grobes Pflichtenheft“)
Im Englischen „Requirements Specification“ (keine Unterscheidung von Lasten-/Pflichtenheft)

\subsection{Welche Verbesserungspotenziale gibt es}

\subsection{Warum verbessern?}

\subsection{Geplante Erweiterungen des Funktionsumfangs}

\subsection{Interviews mit Stakeholdern}