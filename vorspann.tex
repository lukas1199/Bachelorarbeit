% Leere Titelseite
\pagenumbering{gobble}
\newpage\null\thispagestyle{empty}\newpage

\begin{titlepage}
    \normalsize{Hochschule Hannover, Fakultät IV: Wirtschaft und Informatik

    Bachelorarbeit im Studiengang Wirtschaftsinformatik, Wintersemester 2021/2022} 
    
    \sloppy 
    \textbf{\Large{\\Konzeption, Datenmodellierung und prototypischer Aufbau eines Prozess-Tracking-Tools zur Steuerung und Umsetzungsverfolgung einer S/4HANA Transformation im Vorgehensmodell eines IT-Beratungsunternehmens}}
    \vspace{10cm}
    \normalsize{\\Abgabedatum: 07. Februar 2022 \vspace{1cm}\\Lukas Hampel\\Matrikelnummer: 1481025\\Scharnhorststr. 8\\31785 Hameln\vspace{1cm}\\Erstprüfer: Herr Prof. Dr. Raymond Fleck, Hochschule Hannover\\Zweitprüfer: Herr Michael Bloß, adesso orange AG}
\end{titlepage}


\section*{Sperrvermerk}
Diese Arbeit enthält vertrauliche Daten der adesso orange AG. Veröffentlichungen oder Vervielfältigungen dieser Arbeit – auch auszugsweise – sind ohne ausdrückliche Genehmigung von Herrn Michael Bloß nicht gestattet.
\vspace{1cm}
\\This thesis contains confidential data of the adesso orange AG. It may not be disclosed, published or in any other manner made known - even in extracts - to any third party without the expressed written permission of Mr. Michael Bloß.
\newpage


\pagenumbering{Roman}
\setcounter{page}{3}
\section*{Vorbemerkung}
Aus Gründen der besseren Lesbarkeit wird im Folgenden auf die gleichzeitige Verwendung weiblicher und männlicher Sprachformen verzichtet und das generische Maskulinum verwendet. Sämtliche Personenbezeichnungen gelten gleichermaßen für alle Geschlechter.

\newpage

\tableofcontents

\newpage

\section*{Glossar / Abkürzungsverzeichnis}
\begin{xltabular}{\textwidth}{p{0.4\textwidth}p{0.6\textwidth}}
    BTT & Business Transformation Tracker\\
    SAP & ERP-Software der SAP SE\\
    SAP SE & Softwarehersteller aus Walldorf\\
    SE & Europäische Aktiengesellschaft\\
    ERP & Enterprise Ressource Planning\\
    SAP R/1 oder System RF & Erste Generation des ERP-Systems von SAP\\
    SAP R/2 & Zweite Generation des ERP-Systems von SAP\\
    SAP R/3 & Dritte Generation des ERP-Systems von SAP\\
    S/4HANA & Vierte Generation des ERP-Systems von SAP\\
    HANA & Datenbanktechnologie von SAP\\
    Best-Practices & Bekannte Vorgehensweise mit bestem Ergebnis\\
    Requirements Engineering & Anforderungsspezifikation, Systemanalyse\\
    GUI & Grafische Benutzeroberfläche\\
    UML & Unified Modeling Language\\
    OOA & objektorientierte Analyse\\
    IBM & amerikanischer IT-Konzern\\
    DM & Deutsche Mark\\
    IFRS & Internationale Rechnungslegungsvorschriften\\
    AWS & Amazon Web Services\\
    SAP ECC & SAP ERP Central Component\\
    CRM & Customer Relationship Management\\
    SCM & Supply Chain Management\\
    RAM & Random Access Memory (Arbeitsspeicher)\\
    ACID & Atomicity, Consistency, Isolation, Durability \\
    FIORI & Webanwendungen im SAP-System\\
    On-Premise & Vorort\\
    HEC & HANA Enterprise Cloud\\
    EBITDA & Ergebnis vor Zinsen, Steuern, Abschreibungen\\
    adesso active transformation & Vorgehensmodell der adesso orange AG zur Durchführung von S/4HANA-Transformationen\\
    UX & User Experience\\
    EU-DSGVO & Europäische Datenschutz Grundverordnung\\
    Spreadsheet & Tabellendokument\\
    Governance & Gesetzlicher Ordnungsrahmen\\
    Compliance & Innerbetriebliche Richtlinien\\
    Consultant & Berater\\
    (Software)-Suite & Paket mit mehreren Anwendungen\\
    x86-Architektur & Gängige Systemarchitektur von Desktop-PCs\\
    SSD & Solid State Drive\\
    TBD & To be determined
\end{xltabular}
\newpage

\listoffigures{}
\listoftables{}

\newpage

\section*{Kurzfassung}
\doublespacing
\begin{tabularx}{\textwidth}{p{0.95\textwidth}}
    Lukas Hampel\\
    Konzeption, Datenmodellierung und prototypischer Aufbau eines Prozess-Tracking-Tools zur Steuerung und Umsetzungsverfolgung einer S/4HANA Transformation im Vorgehensmodell eines IT-Beratungsunternehmens\\
    Bacherlorarbeit im Studiengang Wirtschaftsinformatik im Wintersemester 2021/2022\\
    Erstprüfer: Herr Prof. Dr. Raymond Fleck, Hochschule Hannover\\
    Zweitprüfer: Herr Michael Bloß, adesso orange AG\\
\end{tabularx}
\vspace{1cm}
\\In dieser Bachelorarbeit geht es um die Konzeptionierung, Datenmodellierung und prototypischen Aufbau eines Prozess-Tracking-Tools im Vorgehensmodell zu einer SAP S/4HANA-Transformation eines IT-Beratungsunternehmens. Dazu werden zuerst die theoretischen Grundlagen zu ERP-Systemen und Transformationen erklärt und danach die SAP SE, sowie ihre Produkte vorgestellt. Im Anschluss wird das im Titel erwähnte IT-Beratungsunternehmen, sowie das Vorgehensmodell zur S/4HANA-Transformation erläutert und genauer auf das zu konzeptionierene Tool eingangen. Dafür wird der momentane Ist-Zustand analysiert und bewertet, sowie eine Problemstellung formuliert. Im darauffolgenden Kapitel werden die Anforderungen an das zu konzeptionierene Tool mittels einer Umfrage, sowie durch persönliche Gespräche erhoben und im Anschluss genauer spezifiziert. Dazu werden sechs verschiedene Anwendungsfälle erläutert und mit Hilfe von Sequenzdiagrammen dargestellt. Im achten Kapitel erfolgt die Datenmodellierung des Systems mittels UML-Klassendiagrammen und Paketdiagrammen. Im darauffolgenden Kapitel wird der GUI-Prototyp vorgestellt und der Aufbau der Benutzeroberfläche errläutert. Die Arbeit endet mit einem abschließenden Fazit.

\newpage

\pagenumbering{arabic}
\setcounter{page}{1}
\begin{normalsize}
\linespread{1.5}